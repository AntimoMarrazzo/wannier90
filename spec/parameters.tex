\chapter{Parameters}

The parameters for a run are specified in the free-format file
seedname.win. The following parameters define the system and must be
specified

\begin{itemize}
\item[{\bf --}] \verb#num_wann#
\item[{\bf --}] \verb#unit_cell_cart#
\item[{\bf --}] \verb#mp_grid#
\item[{\bf --}] One of \verb#kpoint# and \verb#mp_grid_automatic#
\end{itemize}

Other parameters have default values which will be used unless values
are explicitly specified.



\section{Data}
\label{parameter_data}

\begin{table}
\begin{center}
\begin{tabular}{|c|c|p{6cm}|}
\hline
Keyword & Type & Description \\
        &      &             \\
\hline\hline
\multicolumn{3}{|c|}{System Parameters} \\
\hline
{\sc num\_wann }   & I & Number of Wannier Functions \\
{\sc num\_bands }   & I & Number of bands passed to the code \\
{\sc unit\_cell\_cart }   & P & Unit cell vectors \\
{\sc atoms\_cart }*   & P & Positions of atoms in Cartesian coordinates \\
{\sc atoms\_frac }*   & R & Positions of atoms in lattice vectors \\
{\sc mp\_grid }   & I & Dimensions of the Monkhorst-Pack grid \\
{\sc mp\_grid\_automatic }**   & L & Determine the kpoint automatically \\
{\sc kpoints }**   & R & List of kpoints in the Monkhorst-Pack grid \\
{\sc nshells }   & I & Number of shells in finite difference formula \\
{\sc nwhich }   & I & Which shells to use in finite difference formula \\
{\sc postproc\_setup }   & L & To output the nnkp file \\
{\sc cp\_pp }   & L & CP code post-processing \\
{\sc calc\_only\_a }   & L & Only recalculate the projections \\
{\sc exclude\_bands }   & I & List of bands to exclude from the calculation \\
{\sc restart }   & C & Restart from checkpoint file \\
{\sc wann\_continue }   & L & Restart minimisation from file \\
{\sc iprint }   & I & Output verbosity level \\
{\sc length\_unit }   & S & System of units for lengths \\
{\sc devel\_flag }   & S & Flag for development use \\
\hline
\end{tabular}
\caption[Parameter file keywords controlling system parameters.]
{Parameter file keywords defining the system.  Argument types
are represented by, I for a integer, R for a real number, P for a
physical value, L for a logical value and S for a text string.\\
 {\footnotesize
* {\sc atoms\_cart } and  {\sc atoms\_frac } may not both be defined in the same input file.\\
** {\sc mp\_grid\_automatic } and  {\sc kpoints } may not both be defined in the same input file.
}}
\label{parameter_keywords1}
\end{center}
\end{table}



\begin{table}
\begin{center}
\begin{tabular}{|c|c|p{6cm}|}
\hline
Keyword & Type & Description \\
        &      &             \\
\hline\hline
\multicolumn{3}{|c|}{Overlap Parameters} \\
\hline
{\sc wvfn\_formated }   & L & Read the wavefunctions from a  (un)formatted file  \\
{\sc spin }   & I & Which spin channel to read \\
\hline
\end{tabular}
\caption
{Parameter file keywords controlling the aspects of wavefunction overlap.  Argument types
are represented by, I for a integer, R for a real number, P for a
physical value, L for a logical value and S for a text string.}
\label{parameter_keywords2}
\end{center}
\end{table}



\begin{table}
\begin{center}
\begin{tabular}{|c|c|p{6cm}|}
\hline
Keyword & Type & Description \\
        &      &             \\
\hline\hline
\multicolumn{3}{|c|}{Disentanglement Parameters} \\
\hline
{\sc dis\_win\_min }   & P & Bottom of the outer energy window \\
{\sc dis\_win\_max }   & P & Top of the outer energy window \\
{\sc dis\_froz\_min }   & P & Bottom of the inner (frozen) energy window \\
{\sc dis\_froz\_max }   & P & Top of the inner (frozen) energy window \\
{\sc dis\_num\_iter }   & I & Number of iterations for the minimisation
of $\Omega_{I}$ \\
{\sc dis\_mix\_ratio }   & R & Mixing ratio during the minimisation of $\Omega_{I}$\\
{\sc dis\_conv\_tol }   & R & The convergence tolerance for finding $\Omega_{I}$ \\
{\sc dis\_conv\_window }   & I & The number of iterations over which
convergence of $\Omega_{I}$ is assessed. \\ 
\hline
\end{tabular}
\caption[Parameter file keywords controlling disentanglement parameters.]
{Parameter file keywords controlling the disentanglement.  Argument types
are represented by, I for a integer, R for a real number, P for a
physical value, L for a logical value and S for a text string.}
\label{parameter_keywords4}
\end{center}
\end{table}



\begin{table}
\begin{center}
\begin{tabular}{|c|c|p{6cm}|}
\hline
Keyword & Type & Description \\
        &      &             \\
\hline\hline
\multicolumn{3}{|c|}{Wannierise Parameters} \\
\hline
{\sc wannierise }   & L &  Minimise the spread, $\Omega$ \\
{\sc num\_iter }   & I & Number of iterations for the minimisation
of $\Omega$ \\
{\sc num\_cg\_steps }   & I & During the minimisation
of $\Omega$ the number of Conjugate Gradient steps before resetting to
Steepest Descents \\
{\sc mix\_ratio }   & R &Mixing ratio during the minimisation of $\Omega$  \\
{\sc conv\_tol }   & P &The convergence tolerance for finding $\Omega$  \\
{\sc conv\_window }   & I & The number of iterations over which
convergence of $\Omega$ is assessed \\
{\sc coarse\_num\_iter }   & I & Number of initial coarse iterations
during the minimisation
of $\Omega$ \\
{\sc coarse\_mix\_ratio }   & R & Mixing ratio for the initial, coarse
steps during the minimisation of $\Omega$ \\
{\sc num\_dump\_cycles }   & I & Control frequency of checkpointing \\
{\sc num\_print\_cycles }   & I & Control frequency of printing \\
{\sc guiding\_centres }   & L & Use guiding centres \\
{\sc guiding\_centres\_iter }   & I & Frequency of guiding centres \\
\hline
\end{tabular}
\caption[Parameter file keywords controlling the Wannierise routine.]
{Parameter file keywords controlling the wannierisation.  Argument types
are represented by, I for a integer, R for a real number, P for a
physical value, L for a logical value and S for a text string.}
\label{parameter_keywords5}
\end{center}
\end{table}



\begin{table}
\begin{center}
\begin{tabular}{|c|c|p{6cm}|}
\hline
Keyword & Type & Description \\
        &      &             \\
\hline\hline
\multicolumn{3}{|c|}{Plot Parameters} \\
\hline
{\sc wannier\_plot }   & L & Plot the Wannier Functions \\
{\sc wannier\_plot\_supercell }   & I & Size of the supercell for
plotting the Wannier Functions \\
{\sc wannier\_plot\_format }   & S & File format in which to plot the
Wannier Functions \\
{\sc bands\_plot }   & L & Plot and interpolated band structure \\
{\sc kpoint\_path }   & P & K-point path for the interpolated band structure  \\
{\sc bands\_num\_points }   & I & Number of points along the first
section of the k-point path \\
{\sc bands\_plot\_format }   & S & File format in which to plot the
interpolated bands \\
{\sc fermi\_surface\_plot }   & L & Plot the Fermi surface \\
{\sc fermi\_surface\_num\_points }   & I & Number of points in the Fermi
surface plot\\
{\sc fermi\_energy }   & P & The Fermi energy \\
{\sc fermi\_surface\_plot\_format }   & S & File format for the Fermi
surface plot \\
{\sc slice\_plot }   & L & Plot the Wannier Functions along a slice \\
{\sc slice\_coord }   & P & Coordinates of the slice \\
{\sc slice\_num\_points }   & I & Number of points in the slice plot \\
{\sc slice\_plot\_format }   & S & File format of the slice plot \\
{\sc dos\_plot }   & L & Plot the interpolated density of states \\
{\sc dos\_num\_points }   & I & Number of points in the dos plot \\
{\sc dos\_energy\_step }   & P & Size of the energy step in the dos plot \\
{\sc dos\_gaussian\_width }   & P & Width of the convolving gaussian
smearing for the dos plot \\
{\sc dos\_plot\_format }   & S & Format of the dos plot \\
\hline
\end{tabular}
\caption[Parameter file keywords controlling plotting.]
{Parameter file keywords controlling the  plotting.  Argument types
are represented by, I for a integer, R for a real number, P for a
physical value, L for a logical value and S for a text string.}
\label{parameter_keywords6}
\end{center}
\end{table}

\clearpage


\section{System}

\subsection[num\_wann]{\tt integer :: num\_wann}
Number of wannier functions to be found.

No default

\subsection[num\_bands]{\tt integer :: num\_bands}

Default \verb#num_bands#=\verb#num_wann#

\subsection[Cell Lattice Vectors]{Cell Lattice Vectors}

The cell lattice vectors may be specified in Cartesian coordinates.


\noindent \verb#begin unit_cell_cart# \\
$$
\begin{array}{ccc}
R_{1x} & R_{1y} & R_{1z} \\
R_{2x} & R_{2y} & R_{2z} \\
R_{3x} & R_{3y} & R_{3z}
\end{array}
$$
\verb#end unit_cell_cart#

Here $R_{1x}$ is the x-component of the first lattice vector,
$R_{2y}$ is the y-component of the second lattice vector etc.

There is no default


\subsection[Ionic Positions]{Ionic Positions}

The ionic positions may be specified in fractional coordinates relative
to the lattice vectors of the unit cell, or in absolute coordinates.
Only one of \verb#atoms_cart# and \verb#atoms_frac# may be given an input
file.

\subsubsection{atoms\_cart}
 
\noindent \verb#begin atoms_cart#
$$
\begin{array}{cccc}
X  & R_{1i} & R_{1j} & R_{1k} \\
Y  & R_{2i} & R_{2j} & R_{2k} \\
\vdots
\end{array}
$$
\verb#end atoms_cart#


The first entry on a line is the atomic symbol. The next three entries
are the atom's position in Cartesian coordinates in units specified by
\verb#length_unit#.

\subsubsection{atoms\_frac}

\noindent \verb#begin atoms_frac#
$$
\begin{array}{cccc}
X  & R_{1i} & R_{1j} & R_{1k} \\
Y  & R_{2i} & R_{2j} & R_{2k} \\
\vdots
\end{array}
$$
\verb#end atoms_frac#

The first entry on a line is the Atomic symbol. The next three entries
are the atom's position in fractional coordinates.


\subsection[mp\_grid]{\tt integer, dimension :: mp\_grid(3)}
Dimensions of the Monkhorst-Pack mesh

No default

\subsection[mp\_grid\_automatic]{\tt logical :: mp\_grid\_automatic}
\red{Not yet implemented}

If
$\verb#mp_grid_automatic#=\verb#TRUE#$
then a "standard" Monkhorst-Pack grid over the interval (0,1] with dimensions \verb#mp_grid#
will be used. The kpoints will be assumed to be numbered such that the
loop over the x is fastest eg

$$
\begin{array}{cccc}
Kpoint 1 &  0.0 & 0.0& 0.0 \\
Kpoint 2 & 0.25 &0.0 & 0.0 \\
Kpoint 3 & 0.50 &0.0 & 0.0 \\
Kpoint 4 & 0.75 &0.0 & 0.0 \\
Kpoint 5 & 0.0  &0.25& 0.0 \\
\vdots
\end{array}
$$

If $\verb#mp_grid_automatic#=\verb#TRUE#$ then a \verb#kpoint# block must not be present.


{\it This keyword is helpful if one is using a dense MP mesh (eg
  12x12x12) as it saves typing in a very long list of kpoints}

The default for this keyword is \verb#FALSE#.

\subsection[Kpoints]{Kpoints}
Each line gives the coordinate of a k-point
in relative units, i.e. in units of the reciprocal lattice
vectors.
The position  of each k-point in this
list assigns its numbering; the first k-point is k-point 1, the second
is k-point 2, and so on.


\noindent \verb#begin kpoints# \\
$$
\begin{array}{ccc}
 R_{1i} & R_{1j} & R_{1k} \\
 R_{2i} & R_{2j} & R_{2k} \\
\vdots
\end{array}
$$
\verb#end kpoints#

If a kpoint list is specified then \verb#mp_grid_automatic# must be
\verb#FALSE#.

There is no default


\subsection{Shells}

{\it Eventually these keywords will be replaced when we write a better
  kmesh module. But for now we preserve the behavior of the f77 code}



\subsection[nshells]{\tt integer :: nshells}

\subsection[nwhich]{\tt integer :: nwhich}


From wannier.README
 NSHELLS, WHICHSHELLS(NSHELLS). One integer, plus a vector
of integers, as many as specified by the first integer. This is helpful to
deal with tricky geometries. The first integer tells how many shells of
neighbours are used to take derivatives (1, in a cubic cell - 6
neighbours in the first shell, by the way - but as many as 3, say, in
an orthorhombic cell, where every shell has only two neighbours). The
following vector WHICHSHELLS tells the code  which shells to take, once
they are ordered by increasing llength. To clarify ideas, suppose we have an
orthorhombic cell. Now, every shell has 2 k-points; if the orthorhombic
cell is close to cubic, the lowest 3 shells will be one in the x
direction, one in the y direction, and one in the z. But if one
dimension is very different from the others, we might have to skip a
few shells before getting the one we need. If NSHELLS is set to 0,
than the code assumes the general 'trigonal' approach to calculating
derivatives, i.e. using neighbouring k-points along the
directions G\_1, G\_2, G\_3, and G\_1 + isign*G\_2, G\_2 + isign G\_3, and
G\_3 + isign G\_1, where G\_i are the primitive vectors of the
reciprocal lattice, and ISIGN (+1 or -1) is given in the following
line (NOTE: other more complex combinations of G\_i's could be used).

Is it sufficient to specify the bravais lattice? Should we also provide
the option to specify nshells and whichshells instead.


\subsection[postproc\_setup]{\tt logical :: postproc\_setup}
If \verb#postproc_setup#=\verb#TRUE# then the wannier code will write 
a $<$seedname$>$.nnkp file and exit.

The default value is \verb#FALSE#.


\subsection[cp\_pp]{\tt logical :: cp\_pp}
If \verb#cp_pp#=\verb#TRUE# we are using input files from the CP code.
                                                                                                                              
The default value is \verb#FALSE#.


\subsection[iprint]{\tt integer :: iprint}

This indicates the level of verbosity of the output from 0,
the bare minimum to 3, which corresponds to full debugging output. So at
$\verb#iprint#=3$ we might output the spread after every iteration while
at $\verb#iprint#=1$ we might output the spread once every hundred iterations.

The default value of this parameter is 1.

\subsection[length\_unit]{\tt character(len=20) :: length\_unit}
The energy unit to be used for all input and output.

The valid options for this parameter are:
\begin{itemize}
\item[{\bf --}]  Ang
\item[{\bf --}]  Bohr
\end{itemize}

 The default value of this parameter is $\verb#Ang#$

\subsection[devel\_flag]{\tt character(len=50) :: devel\_flag}

Not a parameter for users. Its purpose is to allow a developer to pass a
string into the code to be used inside a new routine as it is developed.

No default

\subsection[calc\_only\_A]{\tt logical :: calc\_only\_A}
\red{Not yet implemented}

If $\verb#calc_only_A#=\verb#.true.#$, then the \textit{ab initio}
code, eg \textsc{pwscf},
calculates only $A_{mn}^{(\mathbf{k})}$. Otherwise, both
$M_{mn}^{(\mathbf{k,b})}$ and $A_{mn}^{(\mathbf{k})}$ are
calculated.

The default value of this parameter is \verb#FALSE#


\subsection[exclude\_bands]{\tt integer :: exclude\_bands}
\red{Not yet implemented}

Exclude states from the calculation of the overlap matrices; for example to select only valence states, or exclude semi-core states

\subsection[restart]{\tt character(len=20) :: restart}

If \verb#restart# is present the code will attempt to restart the calculation
from the $<$seedname$>$.chk file. The value of the parameter
determines the position of the restart

The valid options for this parameter are:
\begin{itemize}
\item[{\bf --}]  \verb#default#. Restart from the point at which the
  check file was written  
\item[{\bf --}]  \verb#wannierise#. Restart from the beginning of the
  wannierise routine 
\item[{\bf --}]  \verb#plot#. Go directly to the plotting phase 


\end{itemize}

\subsection[wann\_continue]{\tt logical :: wann\_continue}

If \verb#wann_continue#=\verb#TRUE# then the code will restart the minimisation
of $\Omega$ from the $<$seedname$>$.dat file when it enters the wannierise routine.

The default is \verb#FALSE#  



\section{Overlaps}



\subsection[wvfn\_formated]{\tt character(len=20) :: wvfn\_formatted}

If $\verb#wvfn_formatted#=TRUE$ the wavefunctions will be read from disk
in formatted (ie ASCII) style. Otherwise they will be read as unformatted
files. Unformatted in generally preferable as the files will take less disk
space and unformatted I/O is significantly faster. However such files
will not be transferable between all machine architectures and formatted
files should be used if transferability is required (ie for test cases).

The default value of this parameter is $\verb#unformatted#$


\subsection[spin]{\tt integer :: spin}
 Value 1 (spin up) or 2 (spin
down). For bands from a spin polarised calculation determines which set
of bands to read in.

The default value of this parameter is 1.





\section{Projection}

See Section \ref{sec:proj}



\section{Disentanglement}
These keywords control the disentanglement routine of SMV. This routine
will be activated if \verb#num_wann#$<$\verb#num_bands#.


\subsection[dis\_win\_min]{\tt real(kind=dp) :: dis\_win\_min}
The lower bound of the outer energy window for the disentanglement
procedure.

The default is the lowest eigenvalue in the system.

\subsection[dis\_win\_max]{\tt real(kind=dp) :: dis\_win\_max}
The upper bound of the outer energy window for the disentanglement
procedure.

The default is the highest eigenvalue in the given states (ie all states
are included in the disentanglement procedure).

\subsection[dis\_froz\_min]{\tt real(kind=dp) :: dis\_froz\_min}
The lower bound of the inner energy window for the disentanglement
procedure. If \verb#dis_froz_min# is specified then \verb#dis_froz_max#
must also be given. If neither are given there are no frozen states.


No default

\subsection[dis\_froz\_max]{\tt real(kind=dp) :: dis\_froz\_max}
The upper bound of the inner energy window for the disentanglement
procedure. If \verb#dis_froz_max# is specified then \verb#dis_froz_min#
must also be given. If neither are given there are no frozen states.

No default

\subsection[dis\_num\_iter]{\tt integer :: dis\_num\_iter}
In the disentanglement procedure, the
number of iterations used to extract the most connected subspace.

The default value is 100.

\subsection[dis\_mix\_ratio]{\tt real(kind=dp) :: dis\_mix\_ratio}
In the disentanglement procedure the mixing parameter to use for
convergence.

The default value is 1.0

\subsection[dis\_conv\_tol]{\tt real(kind=dp) :: dis\_conv\_tol}
\red{Not yet implemented}

In the disentanglement procedure the minimisation is said to to converged
if the fraction change in the spread between successive
iterations is less than
\verb#dis_conv_tol# for \verb#dis_conv_window# iterations.

The default value of this parameter is 1.0E-5 (think a bit about this value)


\subsection[dis\_conv\_window]{\tt integer :: dis\_conv\_window}
\red{Not yet implemented}

In the disentanglement procedure the minimisation is said to to converged
if the fraction change in the spread between successive
iterations is less than
\verb#dis_conv_tol# for \verb#dis_conv_window# iterations.

The default value of this parameter is 3.


\subsection{Other keywords}
 We could have an option to take the lowest N states, rather than using
 an energy window.



\section{Wannerise}
Minimise the non-invariant part of the spread functional.

\subsection[wannierise]{\tt logical :: wannierise}
If $\verb#wannierise#=\verb#TRUE#$ the the non-invariant part of the
spread will be minimise (ie to obtain maximally localised Wannier functions.
The default value of this parameter is $\verb#TRUE#$


\subsection[num\_iter]{\tt integer :: num\_iter}

Total number of iterations in the localization procedure.

The default value is 1000.

\subsection[num\_cg\_steps]{\tt integer :: num\_cg\_steps}

Number of conjugate gradient steps to take before resetting to steepest descents.

The default value is 0.



\subsection[mix\_ratio]{\tt real(kind=dp) :: mix\_ratio}
The mixing parameter to use for convergence.

The default value is 1.0

\subsection[conv\_tol]{\tt real(kind=dp) :: conv\_tol}
\red{Not yet implemented}

The minimisation is said to have converged if the change in
the spread is less than \verb#conv_tol# for \verb#conv_window# iterations.
 If $ \verb#conv_tol#<0$ do not use a convergence criterion, simply run
 for $\verb#num_iter#$ steps. 


The default value of this parameter is -1 (ie preserve the behavior of
the f77 code)


\subsection[conv\_window]{\tt integer :: conv\_window}
\red{Not yet implemented}

The minimisation is said to have converged if the change in
the spread is less than \verb#conv_tol# for \verb#conv_window# iterations.

\verb#conv_tol# for \verb#conv_window# iterations.

The default value of this parameter is 5.


\subsection[coarse\_num\_iter]{\tt integer :: coarse\_num\_iter}

Number of initial coarse steps in the  localization
procedure.

The default value is 10.


\subsection[coarse\_mix\_ratio]{\tt real(kind=dp) :: coarse\_mix\_ratio}
The mixing parameter during the initial coarse steps in the localization
procedure.

The default value is 1.0

\subsection[num\_dump\_cycles]{\tt integer :: num\_dump\_cycles}
Write sufficient information to do a restart every
\verb#num_dump_cycles# iterations.

The default is 0 (ie don't write out any restart information).

\subsection[num\_print\_cycles]{\tt integer :: num\_print\_cycles}
Write data to the $<$seedname$>$.wout file every
\verb#num_print_cycles# iterations.
                                                                                                                              
The default is 1.
                                                                                                                              



\subsection[guiding\_centres]{\tt logical :: guiding\_centres}
Use guiding centres during the minimisation, in order to avoid
local minima.

The default value of this parameter is FALSE

\subsection[guiding\_centres\_iter]{\tt integer :: guiding\_centres\_iter}

If \verb#guiding_centres# is set to true the
guiding centres are used only after \verb#guiding_centres_iter# minimization iterations
have been completed.

The default value of this parameter is 50.




\section{Post-Processing}

 Capabilities:

\begin{itemize}
\item[{\bf --}]  Plot the wannier orbitals			     
\item[{\bf --}]  Plot the wannier functions along an arbitrary plane  
\item[{\bf --}]  Plot the interpolated band structure 		     
\item[{\bf --}]  Plot the Fermi surface 			     
\item[{\bf --}]  Plot the density of states.
\end{itemize}


\subsection[wannier\_plot]{\tt logical :: wannier\_plot}
\red{Not yet implemented}

If $\verb#wannier_plot#=\verb#TRUE#$ the code will write out the
wannier functions in a super-cell \verb#wannier_plot_supercell# times
the original unit cell in a format specified by \verb#wannier_plot_format#

The default value of this parameter is FALSE

\subsection[wannier\_plot\_supercell]{\tt integer :: wannier\_plot\_supercell}
\red{Not yet implemented}

Dimension of the ``super-unit-cell'' in which the Wannier Functions are plotted.
     The super-unit-cell is \verb#wannier_plot_supercell# times the unit cell along all three
     linear dimensions (the 'home' unit cell is kept approximately
     in the middle)


\subsection[wannier\_plot\_format]{\tt character(len=20) :: wannier\_plot\_format}
\red{Not yet implemented}

The valid options for this parameter are:
\begin{itemize}
\item[{\bf --}] xcrysden
\item[{\bf --}] gopenmol
\item[{\bf --}] dan
\end{itemize}

The default value for \verb#wannier_plot_format# is xcrysden.



\subsection[bands\_plot]{\tt logical :: bands\_plot}

If $\verb#bands_plot#=\verb#TRUE#$ the code will calculate the interpolated band structure along
the path in k-space defined by \verb#bands_kpath# using \verb#bands_num_points# along the first
section of the path and write out an output file in a format specified
by \verb#bands_plot_format#. 

The default value of this parameter is FALSE


\subsection[kpoint\_path]{kpoint\_path}
Start and end kpoints, with labels, for sections of the interpolated
bandstructure plot.

\noindent  \verb#begin kpoint_path#
$$
\begin{array}{cccccccc}
G & 0.0 & 0.0 & 0.0 & L & 0.0 & 0.0 & 1.0 \\
L & 0.0 & 0.0 & 1.0 & N & 0.0 & 1.0 & 1.0 \\
\vdots
\end{array}
$$
\verb#end kpoint_path#

There is no default

\subsection[bands\_num\_points]{\tt integer :: bands\_num\_points}

If $\verb#bands_plot#=\verb#TRUE#$ the number of points along the first
section of the bandstructure plot given by \verb#kpoint_path#. Other
sections will have the same density of kpoints.

The default value for \verb#bands_num_points# is 100


\subsection[bands\_plot\_format]{\tt character(len=20) :: bands\_plot\_format}

Format in which to plot the interpolated band structure

The valid options for this parameter are:
\begin{itemize}
\item[{\bf --}] gnuplot
\item[{\bf --}] xmgrace (possibly)
\end{itemize}

The default value for \verb#bands_plot_format# is gnuplot.


\subsection[fermi\_surface\_plot]{\tt logical :: fermi\_surface\_plot}

If $\verb#fermi_surface_plot#=\verb#TRUE#$ the code will calculate,
through Wannier interpolation, the
eigenvalues on a regular grid with \verb#fermi_surface_num_points# in
each direction. The code will write a file in bxsf format which can be
read with Xcrysden and used to plot the Fermi surface.

The default value of this parameter is FALSE


\subsection[fermi\_surface\_num\_points]{\tt integer :: fermi\_surface\_num\_points}

If $\verb#fermi_surface_plot#=\verb#TRUE#$ the number of divisions in
the regular kpoint grid used to calculate the fermi surface.

The default value for \verb#fermi_surface_num_points# is 50


\subsection[fermi\_energy]{\tt real(kind=dp) :: fermi\_energy}
The Fermi energyi eV. Whilst this is not directly used by the Wannier 
code is a useful parameter to set for Fermi surface plots as
it will be written into the bxsf file.
i
The default value is 0.0eV


\subsection[fermi\_surface\_plot\_format]{\tt character(len=20) :: fermi\_surface\_plot\_format}

Format in which to plot the Fermi surface. I don't know of any other ways
to plot Fermi surface other than bxsf/Xcrysden. But we have this
parameter for future developments. 

The valid options for this parameter are:
\begin{itemize}
\item[{\bf --}] bxsf
\end{itemize}

The default value for \verb#fermi_surface_plot_format# is bxsf.


\subsection[slice\_plot]{\tt logical :: slice\_plot}
\red{Not yet implemented}

If $\verb#slice_plot#=\verb#TRUE#$ plot the wannier orbitals along
 slices in the super-unit-cell defined by \verb#slice_coord#.




The default value of this parameter is FALSE

\subsection[slice\_coord]{slice\_coord}
\red{Not yet implemented}

\noindent \verb#begin slice_coord#
$$
\begin{array}{ccccccccc}
O_x & O_y & O_x & X_x & X_y & X_z & Y_x & Y_y & Y_z \\
\vdots
\end{array}
$$
\verb#end slice_coord#

Define the direction of the plotting slices. O is the origin. The slice
is defined by the lines OX and OY.

There is no default

\subsection[slice\_num\_points]{\tt integer :: slice\_num\_points}
\red{Not yet implemented}

If $\verb#slice_plot#=\verb#TRUE#$ the number of points in the first
direction of the slice. The number of points in the second direction
will be chosen to give the same density of points.

The default value for \verb#slice_num_points# is 50


\subsection[slice\_plot\_format]{\tt character(len=20) :: slice\_plot\_format}
\red{Not yet implemented}

Format in which to plot the interpolated bandstructure

The valid options for this parameter are:
\begin{itemize}
\item[{\bf --}] gnuplot
\item[{\bf --}] plotmtv
\end{itemize}

The default value for \verb#slice_plot_format# is gnuplot.




\subsection[dos\_plot]{\tt logical :: dos\_plot}
\red{Not yet implemented}

If $\verb#dos_plot#=\verb#TRUE#$ the code will calculate,
through Wannier interpolation, the
eigenvalues on a regular grid with dimension \verb#dos_num_points#. The
density of states will be calculated by applying a Gaussian smearing of
width \verb#dos_gaussian_width#.

The default value of this parameter is FALSE


\subsection[dos\_num\_points]{\tt integer :: dos\_num\_points}
\red{Not yet implemented}

If $\verb#dos_plot#=\verb#TRUE#$ the dimension of the kpoint mesh
to sample in calculating the density of states.

The default value for \verb#dos_num_points# is 50

\subsection[dos\_energy\_step]{\tt real(kind=dp) :: dos\_energy\_step}
\red{Not yet implemented}

The density of states will be calculated from the
lowest to highest eigenenergies in the system. \verb#dos_energy_step# determines
the size of the steps on the energy axis.

The default value for \verb#dos_energy_step# is 0.02eV

\subsection[dos\_gaussian\_width]{\tt real(kind=dp) :: dos\_gaussian\_width}
\red{Not yet implemented}

The width of the gaussian smearing to apply to each eigenenergy when
calculating the dos.


The default value for \verb#dos_gaussian_width# is 0.2eV

\subsection[dos\_plot\_format]{\tt character(len=20) :: dos\_plot\_format}
\red{Not yet implemented}

Format in which to plot the density of states

The valid options for this parameter are:
\begin{itemize}
\item[{\bf --}] gnuplot
\item[{\bf --}] xmgrace
\end{itemize}

The default value for \verb#dos_plot_format# is gnuplot.

