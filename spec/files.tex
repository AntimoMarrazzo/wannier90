\chapter{Files}


\section{seedname.win}
INPUT. The master input file; contains the specification of the system
and any parameters for the run. 

\subsection{Units}

The following are the dimensional quantities that are
specified in the master input file:

\begin{itemize}
\item Direct lattice vectors
\item Positions (of atomic or projection) centres in real space
\item Energy windows
\item Positions of $\mathbf{k}$-points in reciprocal space
\item \verb#zona# and \verb#box-size# (see Section~\ref{sec:proj})
\end{itemize}

Notes:

\begin{itemize}
\item The units (either \verb#ang#
  (default) or \verb#bohr#) can be set in the first line of the blocks \verb#unit_cell_cart# and \verb#atoms_cart#
\item Energy is always in eV.
\item Positions of $\mathbf{k}$-points are always in crystallographic
  coordinates
relative to the reciprocal lattice vectors.
\item \verb#box-size# and \verb#zona# always in Angstrom and
  reciprocal Angstrom, respectively
\item The keyword \verb#length_unit# may be set to \verb#ang#
  (default) or \verb#bohr#, in order to set the units in which the
  quantities in the output file are written.
\end{itemize}

The reciprocal lattice vectors
$\{\mathbf{B}_{1},\mathbf{B}_{2},\mathbf{B}_{3}\}$ are defined in
terms
of the direct lattice vectors
$\{\mathbf{A}_{1},\mathbf{A}_{2},\mathbf{A}_{3}\}$ by the equation

\begin{equation}
\mathbf{B}_{1} = \frac{2\pi}{\Omega}\mathbf{A}_{2}\times\mathbf{A}_{3}
\ \ \ \mathrm{etc.},
\end{equation}

where the cell volume is
$\Omega=\mathbf{A}_{1}\cdot(\mathbf{A}_{2}\times\mathbf{A}_{3})$.

\section{seedname.mmn}
INPUT. See Chapter~\ref{ch:wann-pp}.

\section{seedname.amn}
INPUT. See Chapter~\ref{ch:wann-pp}.

\section{seedname.eig}
INPUT. See Chapter~\ref{ch:wann-pp}.

\section{seedname.nnkp} \label{sec:old-nnkp}
OUTPUT. See Chapter~\ref{ch:wann-pp}.

\section{seedname.wout}
OUTPUT. The master output file.


\section{seedname.chk}
INPUT/OUTPUT. Sufficient information to restart the calculation or enter the
plotting phase.

\section{UNKp.s}
INPUT. Read if \verb#wannier_plot#=\verb#TRUE# and used to plot the
Wannier functions.

The periodic part of the bloch states represented on a regular real
 space grid, indexed by k-point \verb#p# (from 1 to \verb#num_kpts#)
 and spin \verb#s# ('1' for 'up', '2' for 'down').

The name of the wavefunction file is assumed to have the form:

\begin{verbatim}
    write(wfnname,200) p,spin
200 format ('UNK',i5.5,'.',i1)
\end{verbatim}

The first line of each file should contain 5 integers: the number of
 grid points in each direction (\verb#ngx#, \verb#ngy# and
 \verb#ngz#), the k-point number \verb#ik# and the total number of
 bands \verb#num\_band# in the file. The full file will be read by Wannier90 as:

\begin{verbatim}
read(file_unit) ngx,ngy,ngz,ik,nbnd
do loop_b=1,num_bands
  read(file_unit) (r_wvfn(nx,loop_b),nx=1,ngx*ngy*ngz)
end do
\end{verbatim}

The file can be in formatted or unformatted style, this is controlled
by the logical keyword \verb#wvfn_formatted#. 



