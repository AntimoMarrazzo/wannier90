\chapter{Frequently Asked Questions}\label{chap:faq}


\section{General Questions}

\subsection{What is \wannier?}

\wannier\ is a computer package, written in Fortran90, for obtaining
maximally-localised Wannier functions, using them to calculate
bandstructures, Fermi surfaces, dielectric properties, sparse
Hamiltonians and many things besides.

\subsection{Where can I get \wannier?}

The most recent release of \wannier\ is always available on our
website {\tt www.wannier.org}.

\subsection{Where can I get the most recent information about
  \wannier?}

The latest news about \wannier\ can be followed on our website {\tt
  www.wannier.org}.

\subsection{Is \wannier\ free?}

Yes! \wannier\ is available for use free-of-charge under the GNU
General Public Licence. See the file {\tt LICENCE} that comes with the
\wannier\ distribution or the GNU hopepage at {\tt www.gnu.org}. 

\subsection{Who wrote \wannier?}

\wannier\ is written by Arash A. Mostofi (Imperial College London),
Jonathan. R. Yates (University of Cambridge) and Young-Su Lee (Korea
Institute of Science and Technology). \wannier\ is based on algorithms
written in 1996-7 by Nicola Marzari (Massachusetts Institute of
Technology) and David Vanderbilt (Rutgers University), and in 2000-1
by Ivo Souza (University of California at Berkeley), Nicola Marzari
and David Vanderbilt.

The interface to {\sc pwscf} was written by Stefano de Gironcoli
(SISSA, Trieste). 

\section{Getting Help}

\subsection{Is there a Tutorial available for \wannier?}

Yes! The {\tt examples} directory of the \wannier\ distribution
contains input files for a number of tutorial calculations. The {\tt
  doc} directory contains the accompanying tutorial handout. 

\subsection{Where do I get support for \wannier?}

There are a number of options:

\begin{enumerate}
\item The \wannier\ User Guide, available in the {\tt doc} directory of the
  distribution, and from the webpage ({\tt
  www.wannier.org/user\_guide.html})
\item The \wannier\ webpage for the most recent announcements ({\tt
  www.wannier.org})
\item The \wannier\ mailing list (see {\tt www.wannier.org/forum.html})
\end{enumerate}

\subsection{Is there a mailing list for \wannier?}

Yes! You need to register: go to {\tt www.wannier.org/forum.html} and
follow the instructions. 

\section{Providing Help: Finding and Reporting Bugs}

\subsection{I think I found a bug. How do I report it?}

\begin{itemize}
\item Check and double-check. Make sure it's a bug.
\item Check that it is a bug in \wannier\ and not a bug in the
  software interfaced to \wannier.
\item Check that you're using the latest version of \wannier.
\item Send an email to {\tt developers@wannier.org}. Make sure to
  describe the problem and to attach all input and output files
  relating to the problem that you have found.
\end{itemize}

\subsection{I have got an idea! How do I report a wish?}

We're always happy to listen to suggestions. Email your idea to the
  \wannier\ developers at {\tt developers@wannier.org}.

\subsection{I want to help! How can I contribute to \wannier?}

Great! There's always plenty of functionality to add. Email us at {\tt
  developers@wannier.org} to let us know about the functionality you'd
  like to contribute. 

\subsection{I like \wannier! Should I donate anything to its authors?}

Our Swiss bank account number is... just kidding! There is no need to
donate anything, please just cite our paper in any publications that
arise from your use of \wannier:

[ref] A. A. Mostofi, J. R. Yates, Y.-S. Lee, I. Souza, D. Vanderbilt
and N. Marzari, \wannier: A Tool for Obtaining Maximally-Localised
Wannier Functions, {\it Comput. Phys. Commun.}, submitted (2007); {\tt
http://arxiv.org/abs/0708.0650}.

\section{Installation}

\subsection{How do I install \wannier?}

Follow the instructions in the file {\tt README.install} in the main
directory of the \wannier\ distribution.

\subsection{Are there \wannier\ binaries available?}

Not at present.

\subsection{Is there anything else I need?}

Yes. \wannier\ works on top of an electronic structure
calculation. At the time of writing, \wannier\ is interfaced to the
\pwscf\ code, a plane-wave, pseudopotential, density-functional theory
code, which is part of the {\tt quantum-espresso} package. You 
will need to download it from the webpage {\tt
  www.quantum-espresso.org} or {\tt www.pwscf.org}. Then compile \pwscf\
and the \wannier\ interface program {\tt pw2wannier90}. For
instructions, please refer to the
documentation that comes with the {\tt quantum-espresso} distribution.

For examples of how to use \pwscf\ and \wannier\ in conjunction with
each other, see the \wannier\ Tutorial.

Interfaces to other electronic structure codes, such as {\sc
  castep},\footnote{{\tt www.castep.org}}
{\sc fleur}\footnote{{\tt www.flapw.de}} and {\sc
  abinit},\footnote{{\tt www.abinit.org}} are currently in
progress and should become available in the near future. 

\section{Compile-time Problems}

\section{Run-time Problems}

\section{Using \wannier}