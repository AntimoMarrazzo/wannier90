\chapter*{Introduction}
\section*{Getting Help}
The latest version of \wannier\ and documentation can always
be found at \url{http://www.wannier.org}.

There is a \wannier\ mailing list for discussing issues in the
development, theory, coding and algorithms pertinent to MLWF.
You can register for this mailing list by following the links at
\url{http://www.wannier.org/forum.html}

Finally, many frequently asked questions are answered in
Appendix~\ref{chap:faq}. 

\section*{Citation}
We ask that you acknowledge the use of \wannier\ in any publications
arising from the use of this code through the following reference
\begin{quote}
[ref] A.~A.~Mostofi, J.~R.~Yates, G.~Pizzi, Y.-S.~Lee, I.~Souza, D.~Vanderbilt
and N.~Marzari,\\
An updated version of \wannier: 
A Tool for Obtaining Maximally-Localised Wannier
  Functions, {\it Comput. Phys. Commun.} {\bf 185}, 2309 (2014)\\
\url{http://dx.doi.org/10.1016/j.cpc.2014.05.003}

%%\url{http://www.wannier.org/}
\end{quote}                                                              

It would also be appropriate to cite the original articles:\\\\
Maximally localized generalized Wannier functions for composite
  energy bands,\\ 
N. Marzari and D. Vanderbilt, {\it Phys. Rev. B} {\bf 56}, 12847 (1997)\\\\
Maximally localized Wannier functions for entangled energy bands,\\
I. Souza, N. Marzari and D. Vanderbilt, {\it Phys. Rev. B} {\bf 65}, 035109 (2001)


\section*{Credits}
The present release of \wannier\ was written by Arash A. Mostofi
(Imperial College London, UK), Giovanni Pizzi (EPFL, Switzerland), Ivo Souza 
(Universidad del Pais Vasco, Spain) and
Jonathan R. Yates (University of
Oxford, UK). Contributors to the code include:
Young-Su Lee (KIST, S. Korea): specialised Gamma point routines and transport;
Matthew Shelley (Imperial College London, UK): transport;
Nicolas Poilvert (Penn State University, USA): transport;
Raffaello Bianco (Univ. Pierre et Marie Curie Paris 6 and CNRS):  k-slice plotting;
Gabriele Sclauzero (ETH, Zurich, Switzerland): disentanglement in spheres in k-space;
David Strubbe (MIT, USA): various bugfixes/improvements;
Rei Sakuma (Lund University, Sweden): Symmetry-adapted Wannier functions;
Yusuke Nomura (U. Tokyo, JP): Symmetry-adapted Wannier functions, non-collinear spin with ultrasoft in pw2wannier90;
Takashi Koretsune (Riken, JP): Symmetry-adapted Wannier functions, non-collinear spin with ultrasoft in pw2wannier90;
Yoshiro Nohara (Atomic-scale material simulations, Co., Ltd.): Symmetry-adapted Wannier functions;
Ryotaro Arita (Riken, JP): Symmetry-adapted Wannier functions;
Lorenzo Paulatto (UPMC Paris, FR): Improvements to the interpolation routines, non-collinear spin with ultrasoft in pw2wannier90;
Florian Thole (ETHZ, CH): non-collinear spin with ultrasoft in pw2wannier90;
Pablo Garcia Fernandez (Unican, ES): Matrix elements of the position operator;
Dominik Gresch (ETHZ, CH): FORD infrastructure for code documentation;
Samuel Ponce (Oxford University, UK): Test suite for Wannier90;
Marco Gibertini (EPFL, CH): Improvements to the interpolation routines;
Christian Stieger (ETHZ, CH): Routine to print the U matrices;
Stepan Tsirkin (Universidad del Pais Vasco, Spain): bug fixes in the berry module.
 \wannier\ is
based on the Fortran 77 codes written for isolated bands by Nicola Marzari
and David Vanderbilt, for entangled bands by Ivo Souza, Nicola Marzari,
and David Vanderbilt, and for quantum transport by Marco Nardelli.

Acknowledgements: Stefano de Gironcoli (SISSA, Trieste, Italy) for the
\pwscf\ interface, Timo Thonhauser and Graham Lopez (Wake Forest, USA) 
extended this to add terms needed for orbital magnetisation
; Michel Posternak (EPFL, Switzerland) for the
original plotting routines. Daniel Aberg (LLNL, USA) for povray plotting routines, w90vdw is written by
Lampros Andrinopoulos, Nicholas D. M. Hine and Arash A. Mostofi at Imperial College London. 

\wannier\ \copyright\ 2007-2017 Arash A. Mostofi, Jonathan R. Yates,
Young-Su Lee, Giovanni Pizzi, Ivo Souza, David Vanderbilt and Nicola Marzari

\section*{Licence}
All the material in this distribution is free software; you can
redistribute it and/or 
modify it under the terms of the GNU General Public License
as published by the Free Software Foundation; either version 2
of the License, or (at your option) any later version.

This program is distributed in the hope that it will be useful,
but WITHOUT ANY WARRANTY; without even the implied warranty of
MERCHANTABILITY or FITNESS FOR A PARTICULAR PURPOSE.  See the
GNU General Public License for more details.

You should have received a copy of the GNU General Public License
along with this program; if not, write to the Free Software
Foundation, Inc., 51 Franklin Street, Fifth Floor, Boston, MA  02110-1301, USA.


 
