\chapter{Parameters}

\section{Usage}
{\tt
\begin{quote}
wannier90.x [-pp] [seedname]
\end{quote} }
\begin{itemize}
\item{ {\tt seedname} If a seedname string is given the code will read its input
from a file {\tt seedname.win}. The default value is {\tt wannier}.}
\item { {\tt -pp} This optional flag tells the code to generate
a list of the required overlaps and then exit. 
This information is written to the file {\tt seedname.nnkp}.}
\end{itemize}

\section{win File}
The Wannier90 input file {\tt seedname.win} has a flexible free-form structure.

The ordering of the keywords is not significant. Case is ignored (so
\verb#num_bands# is the same as \verb#Num_Bands#). Characters after !, or \#
are treated as comments. Most keywords have a default value which is
used unless the keyword is 
given in the win file. Keywords can be set in any of the following ways
{\tt
\begin{quote}
num\_wann   4

num\_wann = 4

num\_wann : 4
\end{quote} }
A logical keyword can be set to {\tt .true.} using any of the following
strings: {\tt T}, {\tt true}, {\tt .true.}.

 For futher examples see Chapter \ref{chap:input} and the
 the Wannier90 Tutorial.


\section{Keyword List}
\label{parameter_data}

\begin{table}
\begin{center}
\begin{tabular}{|c|c|p{6cm}|}
\hline
Keyword & Type & Description \\
        &      &             \\
\hline\hline
\multicolumn{3}{|c|}{System Parameters} \\
\hline
{\sc num\_wann }   & I & Number of Wannier Functions \\
{\sc num\_bands }   & I & Number of bands passed to the code \\
{\sc unit\_cell\_cart }   & P & Unit cell vectors \\
{\sc atoms\_cart }*   & P & Positions of atoms in Cartesian coordinates \\
{\sc atoms\_frac }*   & R & Positions of atoms in lattice vectors \\
{\sc mp\_grid }   & I & Dimensions of the Monkhorst-Pack grid \\
%{\sc mp\_grid\_automatic }**   & L & Determine the k-point automatically \\
{\sc kpoints }   & R & List of k-points in the Monkhorst-Pack grid \\
{\sc num\_shells }   & I & Number of shells in finite difference formula \\
{\sc shell\_list }   & I & Which shells to use in finite difference formula \\
\hline
\end{tabular}
\caption[Parameter file keywords controlling system parameters.]
{win file keywords defining the system.  Argument types
are represented by, I for a integer, R for a real number, P for a
physical value, L for a logical value and S for a text string.\\
 {\footnotesize
* {\sc atoms\_cart } and  {\sc atoms\_frac } may not both be defined in
the same input file. }} 
\label{parameter_keywords1}
\end{center}
\end{table}


\begin{table}
\begin{center}
\begin{tabular}{|c|c|p{6cm}|}
\hline
Keyword & Type & Description \\
        &      &             \\
\hline\hline
\multicolumn{3}{|c|}{Job Control} \\
\hline
{\sc postproc\_setup }   & L & To output the nnkp file \\
{\sc cp\_pp }   & L & CP code post-processing \\
{\sc calc\_only\_a }   & L & Only recalculate the projections \\
{\sc exclude\_bands }   & I & List of bands to exclude from the calculation \\
{\sc restart }   & C & Restart from checkpoint file \\
{\sc iprint }   & I & Output verbosity level \\
{\sc length\_unit }   & S & System of units to output lengths \\
{\sc wvfn\_formatted }   & L & Read the wavefunctions from a  (un)formatted file  \\
{\sc spin }   & S & Which spin channel to read \\
{\sc devel\_flag }   & S & Flag for development use \\
\hline
\end{tabular}
\caption[win file keywords.]
{win file keywords defining the system.  Argument types
are represented by, I for a integer, R for a real number, P for a
physical value, L for a logical value and S for a text string.\\
}
\label{parameter_keywords2}
\end{center}
\end{table}





\begin{table}
\begin{center}
\begin{tabular}{|c|c|p{6cm}|}
\hline
Keyword & Type & Description \\
        &      &             \\
\hline\hline
\multicolumn{3}{|c|}{Disentanglement Parameters} \\
\hline
{\sc dis\_win\_min }   & P & Bottom of the outer energy window \\
{\sc dis\_win\_max }   & P & Top of the outer energy window \\
{\sc dis\_froz\_min }   & P & Bottom of the inner (frozen) energy window \\
{\sc dis\_froz\_max }   & P & Top of the inner (frozen) energy window \\
{\sc dis\_num\_iter }   & I & Number of iterations for the minimisation
of $\Omega_{I}$ \\
{\sc dis\_mix\_ratio }   & R & Mixing ratio during the minimisation of $\Omega_{I}$\\
{\sc dis\_conv\_tol }   & R & The convergence tolerance for finding $\Omega_{I}$ \\
{\sc dis\_conv\_window }   & I & The number of iterations over which
convergence of $\Omega_{I}$ is assessed. \\ 
\hline
\end{tabular}
\caption[Parameter file keywords controlling disentanglement parameters.]
{win file keywords controlling the disentanglement.  Argument types
are represented by, I for a integer, R for a real number, P for a
physical value, L for a logical value and S for a text string.}
\label{parameter_keywords4}
\end{center}
\end{table}



\begin{table}
\begin{center}
\begin{tabular}{|c|c|p{6cm}|}
\hline
Keyword & Type & Description \\
        &      &             \\
\hline\hline
\multicolumn{3}{|c|}{Wannierise Parameters} \\
\hline
{\sc num\_iter }   & I & Number of iterations for the minimisation
of $\Omega$ \\
{\sc num\_cg\_steps }   & I & During the minimisation
of $\Omega$ the number of Conjugate Gradient steps before resetting to
Steepest Descents \\
{\sc conv\_tol }   & P &The convergence tolerance for finding $\Omega$  \\
{\sc conv\_window }   & I & The number of iterations over which
convergence of $\Omega$ is assessed \\
{\sc num\_dump\_cycles }   & I & Control frequency of check-pointing \\
{\sc num\_print\_cycles }   & I & Control frequency of printing \\
{\sc write\_r2mn }   & L & Write matrix elements of $r^2$ between
Wannier functions to file \\
{\sc guiding\_centres }   & L & Use guiding centres \\
{\sc num\_guide\_cycles }   & I & Frequency of guiding centres \\
{\sc num\_no\_guide\_iter }   & I & The number of iterations
after which guiding centres are used\\
\hline
\end{tabular}
\caption[Parameter file keywords controlling the Wannierise routine.]
{win file keywords controlling the wannierisation.  Argument types
are represented by, I for a integer, R for a real number, P for a
physical value, L for a logical value and S for a text string.}
\label{parameter_keywords5}
\end{center}
\end{table}



\begin{table}
\begin{center}
\begin{tabular}{|c|c|p{6cm}|}
\hline
Keyword & Type & Description \\
        &      &             \\
\hline\hline
\multicolumn{3}{|c|}{Plot Parameters} \\
\hline
{\sc wannier\_plot }   & L & Plot the Wannier Functions \\
{\sc wannier\_plot\_list } & I & List of Wannier Functions to plot \\
{\sc wannier\_plot\_supercell }   & I & Size of the supercell for
plotting the Wannier Functions \\
{\sc wannier\_plot\_format }   & S & File format in which to plot the
Wannier Functions \\
{\sc wannier\_plot\_mode }   & S & Mode in which to plot the
Wannier Functions, molecule or crystal \\ 
{\sc bands\_plot }   & L & Plot and interpolated band structure \\
{\sc kpoint\_path }   & P & K-point path for the interpolated band structure  \\
{\sc bands\_num\_points }   & I & Number of points along the first
section of the k-point path \\
{\sc bands\_plot\_format }   & S & File format in which to plot the
interpolated bands \\
{\sc fermi\_surface\_plot }   & L & Plot the Fermi surface \\
{\sc fermi\_surface\_num\_points }   & I & Number of points in the Fermi
surface plot\\
{\sc fermi\_energy }   & P & The Fermi energy \\
{\sc fermi\_surface\_plot\_format }   & S & File format for the Fermi
surface plot \\
%{\sc slice\_plot }   & L & Plot the Wannier Functions along a slice \\
%{\sc slice\_coord }   & P & Coordinates of the slice \\
%{\sc slice\_num\_points }   & I & Number of points in the slice plot \\
%{\sc slice\_plot\_format }   & S & File format of the slice plot \\
%{\sc dos\_plot }   & L & Plot the interpolated density of states \\
%{\sc dos\_num\_points }   & I & Number of points in the dos plot \\
%{\sc dos\_energy\_step }   & P & Size of the energy step in the dos plot \\
%{\sc dos\_gaussian\_width }   & P & Width of the convolving gaussian
%smearing for the dos plot \\
%{\sc dos\_plot\_format }   & S & Format of the dos plot \\
\hline
\end{tabular}
\caption[Parameter file keywords controlling plotting.]
{win file keywords controlling the  plotting.  Argument types
are represented by, I for a integer, R for a real number, P for a
physical value, L for a logical value and S for a text string.}
\label{parameter_keywords6}
\end{center}
\end{table}

\clearpage


\section{System}

\subsection[num\_wann]{\tt integer :: num\_wann}
Number of wannier functions to be found.

No default.

\subsection[num\_bands]{\tt integer :: num\_bands} 

Total number of bands passed to the code in the {\tt <seedname>.mmn} file.

Default \verb#num_bands#=\verb#num_wann#

\subsection[Cell Lattice Vectors]{Cell Lattice Vectors}

The cell lattice vectors should be specified in Cartesian coordinates.


\noindent \verb#begin unit_cell_cart# \\
\verb#[units]#
$$
\begin{array}{ccc}
R_{1x} & R_{1y} & R_{1z} \\
R_{2x} & R_{2y} & R_{2z} \\
R_{3x} & R_{3y} & R_{3z}
\end{array}
$$
\verb#end unit_cell_cart#

Here $R_{1x}$ is the x-component of the first lattice vector,
$R_{2y}$ is the y-component of the second lattice vector etc.

\verb#[units]# specifies the units in which the lattice vectors are
defined: either \verb#Bohr# or \verb#Ang#. 

The default value is \verb#Ang#.



\subsection[Ionic Positions]{Ionic Positions}

The ionic positions may be specified in fractional coordinates relative
to the lattice vectors of the unit cell, or in absolute cartesian coordinates.
Only one of \verb#atoms_cart# and \verb#atoms_frac# may be given in the input
file.


\subsubsection{atoms\_cart}

\noindent \verb#begin atoms_cart# \\
\verb#[units]#
$$
\begin{array}{cccc}
X  & R_{1i} & R_{1j} & R_{1k} \\
Y  & R_{2i} & R_{2j} & R_{2k} \\
\vdots
\end{array}
$$
\verb#end atoms_cart#


The first entry on a line is the atomic symbol. The next three entries
are the atom's position in Cartesian coordinates in units specified by
\verb#length_unit#.

\verb#[units]# specifies the units in which the lattice vectors are
defined either \verb#bohr# or \verb#ang#. If not present, the default is \AA.

\subsubsection{atoms\_frac}

\noindent \verb#begin atoms_frac#
$$
\begin{array}{cccc}
X  & R_{1i} & R_{1j} & R_{1k} \\
Y  & R_{2i} & R_{2j} & R_{2k} \\
\vdots
\end{array}
$$
\verb#end atoms_frac#

The first entry on a line is the atomic symbol. The next three entries
are the atom's position in fractional coordinates.


\subsection[mp\_grid]{\tt integer, dimension :: mp\_grid(3)}
Dimensions of the regular (Monkhorst-Pack) k-point mesh. For example,
for a $2\times2\times2$ grid:

\verb#mp_grid : 2  2  2#

No default.


%
%\subsection[mp\_grid\_automatic]{\tt logical :: mp\_grid\_automatic}
%\red{Not yet implemented}
%
%If
%$\verb#mp_grid_automatic#=\verb#TRUE#$
%then a "standard" Monkhorst-Pack grid over the interval (0,1] with dimensions \verb#mp_grid#
%will be used. The k-points will be assumed to be numbered such that the
%loop over the x is fastest eg
%
%$$
%\begin{array}{cccc}
%Kpoint 1 &  0.0 & 0.0& 0.0 \\
%Kpoint 2 & 0.25 &0.0 & 0.0 \\
%Kpoint 3 & 0.50 &0.0 & 0.0 \\
%Kpoint 4 & 0.75 &0.0 & 0.0 \\
%Kpoint 5 & 0.0  &0.25& 0.0 \\
%\vdots
%\end{array}
%$$
%
%If $\verb#mp_grid_automatic#=\verb#TRUE#$ then a \verb#kpoint# block must not be present.
%

%{\it This keyword is helpful if one is using a dense MP mesh (eg
%  12x12x12) as it saves typing in a very long list of kpoints}
%
%The default for this keyword is \verb#FALSE#.

\subsection[Kpoints]{Kpoints}
Each line gives the coordinate of a k-point
in relative units, i.e. in units of the reciprocal lattice
vectors.
The position  of each k-point in this
list assigns its numbering; the first k-point is k-point 1, the second
is k-point 2, and so on.


\noindent \verb#begin kpoints# \\
$$
\begin{array}{ccc}
 R_{1i} & R_{1j} & R_{1k} \\
 R_{2i} & R_{2j} & R_{2k} \\
\vdots
\end{array}
$$
\verb#end kpoints#

%If a kpoint list is specified then \verb#mp_grid_automatic# must be
%\verb#FALSE#.

There is no default.

\subsection{Shells}

The Marzari-Vanderbilt scheme requires a finite difference expression
for $\nabla_k$ 
defined on a uniform Monkhorst-Pack mesh of k-points. One choice (the
`B1' condition of MV) 
is to choose shells of k-point neighbours to satisfy the equation
\begin{equation}\nonumber
\sum_s w_s \sum_i b_{i\alpha}^s b_{i\beta}^s = \delta_{\alpha\beta}
\end{equation}
`$s$' indexes the shell, `$i$' indexes k-points in that shell, $w_s$ is a weight factor for the shell s, ${\bf b}_{i}^s$
is a vector connecting a k-point to one of it nearest neighbours, $\alpha$ and $\beta$ are Cartesian coordinates.

\subsection[num\_shells]{\tt integer :: num\_shells}

If \verb#num_shells#$>0$ the number of shells to include in the finite difference expression.
If \verb#num_shells#$=0$ the code will choose the shells automatically. 

The default value is 0.

\subsection[shell\_list]{\tt integer :: shell\_list(num\_shells)}

If \verb#num_shells#$>0$ \verb#shell_list# is vector listing the shells
to include in the finite difference expression.

\section{Projection}

 The projections block defines a set of localised functions used to
 generate an initial guess for the unitary transformations. 
This data will be written in the {\tt <seedname>.nnkp} file.

\noindent \verb#begin projections# \\\\
\verb#end projections#

For details see section \ref{sec:proj}.

\section{Job Control}

\subsection[postproc\_setup]{\tt logical :: postproc\_setup}
If \verb#postproc_setup#=\verb#TRUE# then the wannier code will write 
 {\tt <seedname>.nnkp} file and exit.
If Wannier90 is called with the option {\tt -pp} then 
 \verb#postproc_setup# is set to 
\verb#TRUE#, over-riding its
value in the {\tt <seedname>.win} file.

The default value is \verb#FALSE#.


\subsection[cp\_pp]{\tt logical :: cp\_pp}
If \verb#cp_pp#=\verb#TRUE# we are using input files from the CP code.

The default value is \verb#FALSE#.


\subsection[iprint]{\tt integer :: iprint}

This indicates the level of verbosity of the output from 0,
the bare minimum, to 3, which corresponds to full debugging output.

The default value is 1.

\subsection[length\_unit]{\tt character(len=20) :: length\_unit}
The length unit to be used for output.

The valid options for this parameter are:
\begin{itemize}
\item[{\bf --}]  \verb#Ang# (default)
\item[{\bf --}]  \verb#Bohr#
\end{itemize}

\subsection[devel\_flag]{\tt character(len=50) :: devel\_flag}

Not a regular keyword. Its purpose is to allow a developer to pass a
string into the code to be used inside a new routine as it is developed.

No default.

%\subsection[calc\_only\_A]{\tt logical :: calc\_only\_A}
%\red{Not yet implemented}
%
%If $\verb#calc_only_A#=\verb#.true.#$, then the \textit{ab initio}
%code, eg \textsc{pwscf},
%calculates only $A_{mn}^{(\mathbf{k})}$. Otherwise, both
%$M_{mn}^{(\mathbf{k,b})}$ and $A_{mn}^{(\mathbf{k})}$ are
%calculated.
%
%The default value of this parameter is \verb#FALSE#.


\subsection[exclude\_bands]{\tt integer :: exclude\_bands(:)}

A k-point independent list of states to excluded from the calculation of the overlap matrices;
 for example to select only valence states, or ignore semi-core states.
 This keyword is passed to the first-principles code via the
 {\tt <seedname>.nnkp} file. 

\subsection[restart]{\tt character(len=20) :: restart}

If \verb#restart# is present the code will attempt to restart the calculation
from the {\tt <seedname>.chk } file. The value of the parameter
determines the position of the restart

The valid options for this parameter are:
\begin{itemize}
\item[{\bf --}]  \verb#default#. Restart from the point at which the
  check file was written  
\item[{\bf --}]  \verb#wannierise#. Restart from the beginning of the
  wannierise routine 
\item[{\bf --}]  \verb#plot#. Go directly to the plotting phase 


\end{itemize}



\subsection[wvfn\_formated]{\tt character(len=20) :: wvfn\_formatted}

If \verb#wvfn_formatted#=\verb#TRUE# the wavefunctions will be read from disk
as formatted (ie ASCII) files. Otherwise they will be read as unformatted
files. Unformatted in generally preferable as the files will take less disk
space  I/O is significantly faster. However such files
will not be transferable between all machine architectures and formatted
files should be used if transferability is required (ie for test cases).

The default value of this parameter is $\verb#FALSE#$.


\subsection[spin]{\tt character(len=20) :: spin}
For bands from a spin polarised calculation determines which set
of bands to read in, either \verb#up# or \verb#down#.

The default value of this parameter is \verb#up#.


\section{Disentanglement}
These keywords control the disentanglement routine of SMV. This routine
will be activated if \verb#num_wann#$<$\verb#num_bands#.


\subsection[dis\_win\_min]{\tt real(kind=dp) :: dis\_win\_min}
The lower bound of the outer energy window for the disentanglement
procedure.

The default is the lowest eigenvalue in the system.

\subsection[dis\_win\_max]{\tt real(kind=dp) :: dis\_win\_max}
The upper bound of the outer energy window for the disentanglement
procedure.

The default is the highest eigenvalue in the given states (ie all states
are included in the disentanglement procedure).

\subsection[dis\_froz\_min]{\tt real(kind=dp) :: dis\_froz\_min}
The lower bound of the inner energy window for the disentanglement
procedure. 

If \verb#dis_froz_max# is given the default for 
\verb#dis_froz_min# is \verb#dis_win_min#.


\subsection[dis\_froz\_max]{\tt real(kind=dp) :: dis\_froz\_max}
The upper bound of the inner energy window for the disentanglement
procedure. If \verb#dis_froz_max# is  not specified then 
there are no frozen states.

No default.

\subsection[dis\_num\_iter]{\tt integer :: dis\_num\_iter}
In the disentanglement procedure, the
number of iterations used to extract the most connected subspace.

The default value is 100.

\subsection[dis\_mix\_ratio]{\tt real(kind=dp) :: dis\_mix\_ratio}
In the disentanglement procedure the mixing parameter to use for
convergence.

The default value is 1.0

\subsection[dis\_conv\_tol]{\tt real(kind=dp) :: dis\_conv\_tol}

In the disentanglement procedure the minimisation is said to to converged
if the fractional change in the spread between successive
iterations is less than
\verb#dis_conv_tol# for \verb#dis_conv_window# iterations.

The default value is 1.0E-10


\subsection[dis\_conv\_window]{\tt integer :: dis\_conv\_window}

In the disentanglement procedure the minimisation is said to to converged
if the fractional change in the spread between successive
iterations is less than
\verb#dis_conv_tol# for \verb#dis_conv_window# iterations.

The default value of this parameter is 3.



\section{Wannierise}
Minimise the non-invariant part of the spread functional.

\subsection[num\_iter]{\tt integer :: num\_iter}

Total number of iterations in the minimisation procedure.

The default value is 100.

\subsection[num\_cg\_steps]{\tt integer :: num\_cg\_steps}

Number of conjugate gradient steps to take before resetting to steepest descents.

The default value is 5.

%\subsection[conv\_tol]{\tt real(kind=dp) :: conv\_tol}
%\red{Not yet implemented}


%\subsection[conv\_window]{\tt integer :: conv\_window}
%\red{Not yet implemented}

%\verb#conv_tol# for \verb#conv_window# iterations.

%The default value of this parameter is 5.


\subsection[num\_dump\_cycles]{\tt integer :: num\_dump\_cycles}
Write sufficient information to do a restart every
\verb#num_dump_cycles# iterations.

The default is 0 (ie don't write out any restart information).

\subsection[num\_print\_cycles]{\tt integer :: num\_print\_cycles}
Write data to the {\tt <seedname>.wout} file every
\verb#num_print_cycles# iterations.

The default is 1.

\subsection[write\_r2mn]{\tt logical :: write\_r2mn}

If $\verb#write_r2mn#=\verb#true#$, then the matrix elements
$\langle m|r^2|n\rangle$ (where $m$ and $n$ refer to Wannier functions) are written
to file \verb#seedname.r2mn# at the end of the wannierisation procedure.

The default value of this parameter is \verb#FALSE#.


\subsection[guiding\_centres]{\tt logical :: guiding\_centres}
Use guiding centres during the minimisation, in order to avoid
local minima.

The default value is \verb#FALSE#.

\subsection[num\_guide\_cycles]{\tt integer :: num\_guide\_cycles}
If \verb#guiding_centres# is set to true the
guiding centres are used only every \verb#num_guide_cycles#.

The default value is 1.

\subsection[num\_no\_guide\_iter]{\tt integer :: num\_no\_guide\_iter}
If \verb#guiding_centres# is set to true the
guiding centres are used only after \verb#num_no_guide_iter#
minimisation iterations have been completed.

The default value is 0.


\section{Post-Processing}

 Capabilities:

\begin{itemize}
\item[{\bf --}]  Plot the Wannier functions
\item[{\bf --}]  Plot the interpolated band structure 		     
\item[{\bf --}]  Plot the Fermi surface 			     
%\item[{\bf --}]  Plot the density of states.
\end{itemize}


\subsection[wannier\_plot]{\tt logical :: wannier\_plot}

If $\verb#wannier_plot#=\verb#TRUE#$ the code will write out the
wannier functions in a super-cell \verb#wannier_plot_supercell# times
the original unit cell in a format specified by
\verb#wannier_plot_format#

The default value of this parameter is \verb#FALSE#.

\subsection[wannier\_plot\_supercell]{\tt integer :: wannier\_plot\_supercell}

Dimension of the ``super-unit-cell'' in which the Wannier Functions are plotted.
The super-unit-cell is \verb#wannier_plot_supercell# times the unit cell along all three
linear dimensions (the 'home' unit cell is kept approximately
in the middle)

The default value is 2.

\subsection[wannier\_plot\_format]{\tt character(len=20) :: wannier\_plot\_format}

The valid options for this parameter are:
\begin{itemize}
\item[{\bf --}] \verb#xcrysden# (default)
%\item[{\bf --}] gopenmol
%\item[{\bf --}] dan
\end{itemize}

\subsection[wannier\_plot\_list]{\tt integer :: wannier\_plot\_list(:)}

 A list of Wannier Functions to plot. The Wannier Functions numbered as per the
 {\tt <seedname>.wout} file after the minimisation of the spread.

 The default behaviour is to plot all Wannier Functions.

\subsection[wannier\_plot\_mode]{\tt character(len=20) :: wannier\_plot\_mode}

Choose the mode in which to plot the Wannier functions, either as a molecule
or as a crystal.

The valid options for this parameter are:
\begin{itemize}
\item[{\bf --}] \verb#crystal# (default)
\item[{\bf --}] \verb#molecule# 
\end{itemize}

\subsection[bands\_plot]{\tt logical :: bands\_plot}

If $\verb#bands_plot#=\verb#TRUE#$ the code will calculate the band
structure, through Wannier interpolation,
 along
the path in k-space defined by \verb#bands_kpath# using \verb#bands_num_points# along the first
section of the path and write out an output file in a format specified
by \verb#bands_plot_format#. 

The default value is \verb#FALSE#.


\subsection[kpoint\_path]{kpoint\_path}
Defines the path in kspace along which to calculate the
bandstructure. Each lines gives the start and end points (with labels)
for a section of the path.

\noindent  \verb#begin kpoint_path#
$$
\begin{array}{cccccccc}
G & 0.0 & 0.0 & 0.0 & L & 0.0 & 0.0 & 1.0 \\
L & 0.0 & 0.0 & 1.0 & N & 0.0 & 1.0 & 1.0 \\
\vdots
\end{array}
$$
\verb#end kpoint_path#


There is no default

\subsection[bands\_num\_points]{\tt integer :: bands\_num\_points}

If $\verb#bands_plot#=\verb#TRUE#$ the number of points along the first
section of the bandstructure plot given by \verb#kpoint_path#. Other
sections will have the same density of k-points.

The default value for \verb#bands_num_points# is 100.


\subsection[bands\_plot\_format]{\tt character(len=20) :: bands\_plot\_format}

Format in which to plot the interpolated band structure
The valid options for this parameter are:
\begin{itemize}
\item[{\bf --}] \verb#gnuplot# (default)
%\item[{\bf --}] xmgrace (possibly)
\end{itemize}

\subsection[fermi\_surface\_plot]{\tt logical :: fermi\_surface\_plot}

If $\verb#fermi_surface_plot#=\verb#TRUE#$ the code will calculate,
through Wannier interpolation, the
eigenvalues on a regular grid with \verb#fermi_surface_num_points# in
each direction. The code will write a file in bxsf format which can be
read with Xcrysden and used to plot the Fermi surface.

The default value is \verb#FALSE#.


\subsection[fermi\_surface\_num\_points]{\tt integer :: fermi\_surface\_num\_points}

If $\verb#fermi_surface_plot#=\verb#TRUE#$ the number of divisions in
the regular k-point grid used to calculate the Fermi surface.

The default value for \verb#fermi_surface_num_points# is 50.


\subsection[fermi\_energy]{\tt real(kind=dp) :: fermi\_energy}
The Fermi energy eV. Whilst this is not directly used by the Wannier 
code is a useful parameter to set for Fermi surface plots as
it will be written into the bxsf file.

The default value is 0.0eV.


\subsection[fermi\_surface\_plot\_format]{\tt character(len=20) :: fermi\_surface\_plot\_format}

Format in which to plot the Fermi surface. 
The valid options for this parameter are:
\begin{itemize}
\item[{\bf --}] \verb#xcrysden#  (default)
\end{itemize}


%\subsection[slice\_plot]{\tt logical :: slice\_plot}
%\red{Not yet implemented}
%
%If $\verb#slice_plot#=\verb#TRUE#$ plot the wannier orbitals along
% slices in the super-unit-cell defined by \verb#slice_coord#.




%The default value of this parameter is FALSE
%
%\subsection[slice\_coord]{slice\_coord}
%\red{Not yet implemented}
%
%\noindent \verb#begin slice_coord#
%$$
%\begin{array}{ccccccccc}
%O_x & O_y & O_x & X_x & X_y & X_z & Y_x & Y_y & Y_z \\
%\vdots
%\end{array}
%$$
%\verb#end slice_coord#
%
%Define the direction of the plotting slices. O is the origin. The slice
%is defined by the lines OX and OY.
%
%There is no default
%
%\subsection[slice\_num\_points]{\tt integer :: slice\_num\_points}
%\red{Not yet implemented}
%
%If $\verb#slice_plot#=\verb#TRUE#$ the number of points in the first
%direction of the slice. The number of points in the second direction
%will be chosen to give the same density of points.
%
%The default value for \verb#slice_num_points# is 50
%
%
%\subsection[slice\_plot\_format]{\tt character(len=20) :: slice\_plot\_format}
%\red{Not yet implemented}
%
%Format in which to plot the interpolated bandstructure
%
%The valid options for this parameter are:
%\begin{itemize}
%\item[{\bf --}] gnuplot
%\item[{\bf --}] plotmtv
%\end{itemize}
%
%The default value for \verb#slice_plot_format# is gnuplot.




%\subsection[dos\_plot]{\tt logical :: dos\_plot}
%\red{Not yet implemented}
%
%If $\verb#dos_plot#=\verb#TRUE#$ the code will calculate,
%through Wannier interpolation, the
%eigenvalues on a regular grid with dimension \verb#dos_num_points#. The
%density of states will be calculated by applying a Gaussian smearing of
%width \verb#dos_gaussian_width#.
%
%The default value of this parameter is FALSE
%
%
%\subsection[dos\_num\_points]{\tt integer :: dos\_num\_points}
%\red{Not yet implemented}
%
%If $\verb#dos_plot#=\verb#TRUE#$ the dimension of the k-point mesh
%to sample in calculating the density of states.
%
%The default value for \verb#dos_num_points# is 50
%
%\subsection[dos\_energy\_step]{\tt real(kind=dp) :: dos\_energy\_step}
%\red{Not yet implemented}
%
%The density of states will be calculated from the
%lowest to highest eigenenergies in the system. \verb#dos_energy_step# determines
%the size of the steps on the energy axis.
%
%The default value for \verb#dos_energy_step# is 0.02eV
%
%\subsection[dos\_gaussian\_width]{\tt real(kind=dp) :: dos\_gaussian\_width}
%\red{Not yet implemented}
%
%The width of the gaussian smearing to apply to each eigenenergy when
%calculating the dos.


%The default value for \verb#dos_gaussian_width# is 0.2eV
%
%\subsection[dos\_plot\_format]{\tt character(len=20) :: dos\_plot\_format}
%\red{Not yet implemented}
%
%Format in which to plot the density of states
%
%The valid options for this parameter are:
%\begin{itemize}
%\item[{\bf --}] gnuplot
%\item[{\bf --}] xmgrace
%\end{itemize}

%The default value for \verb#dos_plot_format# is gnuplot.

