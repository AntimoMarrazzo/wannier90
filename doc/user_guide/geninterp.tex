\chapter{Generic Band interpolator}\label{ch:geninterp}

By setting $\verb#geninterp#=\verb#TRUE#$, \postw\ will calculate the
band energies (and possibly the band derivatives, if also
\verb#geninterp_alsofirstder# is set to \verb#TRUE#) on a generic list
of $k$ points provided by the user.

The list of input $k$ points for which the band have to be calculated
is read from the file named {\tt seedname\_geninterp.kpt}. The format
of this file is
described below. The interpolated bands (and velocities) are written on the {\tt
  seedname\_geninterp.dat} file.

A note on parallelization: the routine works correctly for any number of processors,
however it doesn't scale well because all the I/O is done by the root node. In practice,
it is not useful with the present version to use more than one processor.

\section{Files}
\subsection{{\tt seedname\_geninterp.kpt}}
INPUT. Read by \postw\ if {\tt geninterp} is \verb#true#. 

The first line is a comment (its maximum allowed length is 500
characters).

The second line must contain \verb#crystal# (or \verb#rel#) if the
$k$-point coordinates are given in relative (crystallographic) units,
i.e., in fractional units with respect to the primitive reciprocal
lattice vectors.
Otherwise, it must contain \verb#frac# (or \verb#abs#) if instead the
$k-$point coordinates are given in absolute 
coordinates (in units of $2\pi/$\AA) along the $k_x$, $k_y$ and $k_z$
axes.

The third line must contain the number $n$ of following $k$ points.

The following $n$ lines must contain the list of $k$ points in the
format
\begin{verbatim}
kpointidx k1 k2 k3
\end{verbatim}
where \verb#kpointidx# is an integer identifying the given $k$ point,
and \verb#k1#, \verb#k2# and \verb#k3# are the three coordinates of the
$k$ points in the chosen units.


\subsection{{\tt seedname\_geninterp.dat}}
OUTPUT. This file contains the interpolated band energies (and also the band
velocities if the input flag \verb#geninterp_alsofirstder# is \verb#true#).

The first two lines are comments (starting with \#), containing the
comment line as it is read from the input file, and a header.
The following lines contain the band energies (and
derivatives) for each band and $k$ point (the energy
index runs faster than the $k$-point index).
For each of these lines, the first four columns contain the $k$-point index as provided in the
input, and the $k$ coordinates (always in absolute coordinates, in
units of $2\pi/$\AA).
The fifth column contains the band energy.

If \verb#geninterp_alsofirstder# is \verb#true#, three further columns
are printed, containing the three first derivatives of the bands along the $k_x$, $k_y$
and $k_z$ directions.

The $k$ point coordinates are in units of $2\pi/$\AA, the band energy is in eV.

\section{Parameters}
The list of parameters of the Generic Band Interpolation module are summarized in Table~\ref{parameter_keywords_geninterp}. 

\begin{table}
\begin{center}
\begin{tabular}{|c|c|p{6cm}|}
\hline
Keyword & Type & Description \\
        &      &             \\
\hline\hline
\multicolumn{3}{|c|}{Generic Band Interpolation Parameters} \\
\hline
{\sc geninterp}   & L & Calculate bands for given set of $k$ points \\
{\sc geninterp\_alsofirstder} & L & Calculate also first derivatives\\ 
\hline
\end{tabular}
\caption[Parameter file keywords controlling the Generic Band Interpolation module.]
{{\tt seedname.win} file keywords controlling the Generic Band Interpolation module. Argument types
are represented by, I for a integer, R for a real number, P for a
physical value, L for a logical value and S for a text string.}
\label{parameter_keywords_geninterp}
\end{center}
\end{table}

Here follows a description of all parameters.

\subsection[boltzwann]{\tt logical :: geninterp}
Determines whether to enter the Generic Band Interpolation routines.

The default value is \verb#false#.

\subsection[geninterp\_alsofirstder]{\tt logical :: geninterp\_alsofirstder}
Whether to calculate also the first derivatives of the bands at the
given $k$ points.

The default value is \verb#false#.

