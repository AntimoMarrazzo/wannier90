\chapter{Overview of the {\tt gyrotropic} module}
\label{ch:berry}

The {\tt gyrotropic} module of {\tt postw90} is called by setting {\tt
  gyrotropic = true} and choosing one or more of the available options for
{\tt gyrotropic\_task}. The module computes the quantities, studied in 
 \cite{tsirkin-arxiv17}.

\section{{\tt berry\_task=-d0}: the Berry curvature dipole  }

The traceless dimensionless tensor
\begin{equation}
\label{eq:D_ab}
D_{ab}=\int\dk\sum_n
\frac{\partial E_n}{\partial{k_a}}
\Omega_n^b
\left(-\frac{\partial f_0}{\partial E}\right)_{E=E_n},
\end{equation}


\section{{\tt berry\_task=-dw}: the finite-frequency generalization of the Berry curvature dipole  }

\begin{equation}
\label{eq:D-tilde}
\widetilde{D}_{ab}(\ww)=\int\dk\sum_n
\frac{\partial E_n}{\partial{k_a}}\widetilde\Omega^b_n(\ww)
\left(-\frac{\partial f_0}{\partial E}\right)_{E=E_n},
\end{equation}

where $\widetilde{\bm\Omega}_{\kk n}(\ww)$ is a finite-frequency
generalization of the Berry curvature:
%
%
\begin{equation}
\label{eq:curv-w}
\widetilde{\bm\Omega}_{\kk n}(\ww)=-
\sum_m\,\frac{\ww^2_{\kk mn}}{\ww^2_{\kk mn}-\ww^2}
\im\left({\bm A}_{\kk nm}\times{\bm A}_{\kk mn}\right)
\end{equation}
Contrary to the Berry
  curvature, the divergence of $\tilde{\bm\Omega}_{\kk n}(\ww)$ is
  generally nonzero. As a result, $\wt D(\ww)$ 
can have a nonzero trace at finite frequencies, $\tilde{D}_\|\neq-2\tilde{D}_\perp$ in Te.

\section{{\tt berry\_task=-C}: the ohmic  conductivity }

In the constant relaxation-time
approximation  we have
 $\sigma_{ab}/\tau=(2\pi e/\hbar)C_{ab}$,  with
%
\beq
\label{eq:C_ab}
C_{ab}=\frac{e}{h}\int\dk\sum_n\,
\frac{\partial E_n}{\partial{k_a}} \frac{\partial E_n}{\partial{k_b}}
\left(-\frac{\partial f_0}{\partial E}\right)_{E=E_n}
\eeq
a positive quantity with
units of surface current density (A/cm).


\section{{\tt berry\_task=-K}: the kinetic magnetoelectric effect (kME) }

A microscopic theory of the intrinsic kME effect in bulk crystals was
recently developed~\cite{yoda-sr15,zhong-prl16}.  The
 response is described by
\bea
\label{eq:K_ab}
K_{ab}&=&\int\dk\sum_n\frac{\partial E_n}{\partial{k_a}}\nn
&\times&\left[m_n^b+\frac{e}{\hbar}(\mu-E_n)\Omega_n^b \right]
\left(-\frac{\partial f_0}{\partial E}\right)_{E=E_n}.
\eea
%
The first term has the same form as (\ref{eq:D_ab}), but with the Berry
curvature replaced by the intrinsic magnetic moment ${\bm m}_{\kk n}$
of the Bloch electrons.  In addition to the spin moment,
${\bm m}_{\kk n}$ has an orbital component given by~\cite{xiao-rmp10} 
%
\beq
\label{eq:m-orb}
{\bm m}^{\rm orb}_{\kk n}=\frac{e}{2\hbar}\im
\bra{{\bm\partial}_\kk u_{\kk n}}\times
(H_\kk-E_{\kk n})\ket{{\bm\partial}_\kk u_{\kk n}},
\eeq
%
where we chose $e>0$.  The second term in (\ref{eq:K_ab}) is a
  Berry-curvature correction that was included in
  Ref.~\cite{yoda-sr15} but not in Ref.~\cite{zhong-prl16}. It
  vanishes at zero temperature, but can become important in
  semiconductors with thermally-activated carriers. The tensor $K$ has
  units of amperes, and it is symmetry-allowed in all 18 gyrotropic
crystal classes.  
In the semiclassical limit, the symmetric part of $K$ gives an
intraband contribution to natural optical
rotation~\cite{zhong-prl16,ma-prb15}.

\section{{\tt berry\_task=-dos}: the density of states }

\section{{\tt berry\_task=-noa}: the interband contributionto the natural optical activity }

\section{{\tt berry\_task=-spin}: compute also the spin component of NOA and kME }

