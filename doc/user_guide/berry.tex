\chapter{Overview of the {\tt berry} module}
\label{ch:berry}

Several electronic properties of crystals are naturally expressed in
terms of ``Berry-type'' quantities in $k$-space~\cite{xiao-rmp10}.
Three such properties are presently calculated in the {\tt berry}
module using Wannier functions as a basis: anomalous Hall conductivity
(AHC), orbital magnetization, and interband optical conductivity.

\section{Theory}

Let us begin by defining the $k$-space Berry connection
%
$$
{\bf A}_n({\bf k})=\langle u_{n{\bf k}}\vert i\bm{\nabla}_{\bf k}\vert
u_{n{\bf k}}\rangle
$$
%
and the Berry curvature
%
$$
\bm{\Omega}_n({\bf k})=\bm{\nabla}_{\bf k}\times {\bf A}_n({\bf k})=
-{\rm Im}
\langle \bm{\nabla}_{\bf k} u_{n{\bf k}}\vert \times
\vert\bm{\nabla}_{\bf k} u_{n{\bf k}}\rangle.
$$

It is convenient to represent the axial Berry curvature vector as an
antisymmetric tensor: $ \Omega_{n,\alpha\beta}({\bf k})
=\epsilon_{\alpha\beta\gamma} \Omega_{n,\gamma}({\bf k})$, where
$\epsilon_{\alpha\beta\gamma}$ is the Levi-Civita tensor.

The
intrinsic AHC $\sigma^{\rm AH}_{\alpha\beta}=-\sigma^{\rm
  AH}_{\beta\alpha}$ is given by the BZ integral of the curvature
summed over the occupied bands~\cite{xiao-rmp10},
%
$$
\sigma^{\rm AH}_{\alpha\beta}=-\frac{e^2}{\hbar} \int_{\rm
  BZ}\frac{d{\bf k}}{(2\pi)^3}\sum_n\,f_{n{\bf
    k}}\,\Omega_{n,\alpha\beta}({\bf k}),
$$
%
where $f_{n{\bf k}}$ is the Fermi occupancy.

The ground-state orbital magnetization of a crystal is given
by~\cite{xiao-rmp10,ceresoli-prb06}
%
$$
{\bf M}_{\rm orb}=\frac{e}{2\hbar}\int_{\rm BZ}\frac{d{\bf
    k}}{(2\pi)^3}\sum_n\,f_{n{\bf k}}\,
{\rm Im}\,\langle \bm{\nabla}_{\bf k}u_{n{\bf k}}\vert
\times
\left(H_{\bf k}+\varepsilon_{n{\bf k}}-2\varepsilon_F\right)
\vert \bm{\nabla}_{\bf k}u_{n{\bf k}}\rangle,
$$
%
where $\varepsilon_{n{\bf k}}$ and $\varepsilon_F$ are
the band energies and the Fermi energy respectively.

In the independent-particle approximation, the absorptive (Hermitean)
part of the interband optical conductivity can be expressed as
%
$$
\sigma^{\rm H}_{\alpha\beta}(\omega)=\frac{\pi e^2}{\hbar}
\int_{\rm BZ}\frac{d{\bf
    k}}{(2\pi)^3}\sum_{n,m}f_{n{\bf k}}(1-f_{m{\bf k}})\omega_{mn,\bf k}
A_{nm,\alpha}({\bf k})A_{mn,\beta}({\bf k})
\delta(\omega-\omega_{mn,\bf k}),
$$
%
where $\hbar\omega_{mn,\bf k}=\varepsilon_{m{\bf
    k}}-\varepsilon_{n{\bf k}}$ and $ {\bf A}_{nm}({\bf k})=\langle
u_{n{\bf k}}\vert i\bm{\nabla}_{\bf k}\vert u_{m{\bf k}}\rangle$ is a
matrix generalization of the Berry connection (not to be confused with
the projection matrix $A^{({\bf k})}_{mn}$ of Eq.~(1.8)).  This is the
usual Kubo-Greenwood formula, with the interband velocity matrix
elements recast in terms of off-diagonal elements of ${\bf
  A}_{nm}({\bf k})$~\cite{blount}. Its real and imaginary parts are
symmetric and antisymmetric under $\alpha\leftrightarrow\beta$
respectively,
%
$$
\sigma^{\rm H}_{\alpha\beta}(\omega)={\rm Re}\,\sigma^{\rm S}_{\alpha\beta}(\omega)+
i{\rm Im}\, \sigma^{\rm A}_{\alpha\beta}(\omega).
$$
%
The symmetric part is the ``ordinary'' optical conductivity; the
antisymmetric part is the ``dichroic'' component, related to
magneto-optical effects such as magnetic circular dichroism and the
magneto-optical Kerr effect.

Once the Hermitean (absorptive) part is calculated over a wide range
of frequencies, the anti-Hermitean (reactive) part can be extracted
from the Kramers-Kronig relations.

\section{Implementation}

The basic idea of the implementation in the {\tt berry} module is to
calculate the needed Berry-type quantities very efficiently at
arbitrarily points in the BZ by Wannier interpolation, once certain
Wannier matrix elements have been tabulated.

As described in Refs.~\cite{wang-prb06,yates-prb07}, the connection
${\bf A}_{nm}({\bf k})$ and curvature $\bm{\Omega}_n({\bf k})$
entering the optical conductivity and AHC expressions can be
calculated at arbitrary points ${\bf k}$ once the matrix elements
$\langle {\bf 0}n\vert H\vert {\bf R}m\rangle$ and $\langle {\bf
  0}n\vert {\bf r}\vert {\bf R}m\rangle$ of the Hamiltonian and
position operator are known.  Luckily, these matrix elements are
readily available at the end of a standard MLWF calculation with {\tt
  wannier90}. In particular, the position-operator matrix can be
calculated as a Fourier transform of the overlap matrices in
Eq.~(1.7),
%
$$\langle u_{n{\bf k}}\vert u_{m{\bf k}+{\bf b}}\rangle.
$$

The presence of the Hamiltonian sandwiched in the middle of the
orbital magnetization formula complicates things a bit, and further
Wannier matrix elements are needed which are not
available ``for free'' at the end of a standard {\tt wannier90}
run. In order to calculate them by Fourier transforms, one more piece
of information must be passed on from the $k$-space {\it ab-initio}
calculation, namely, the matrix elements
%
$$\langle u_{n{\bf k}+{\bf b}_1}\vert
H_{\bf k}\vert u_{m{\bf k}+{\bf b}_2}\rangle
$$
%
over the {\it ab-initio} $k$-point mesh~\cite{lopez-prb12}.  The
evaluation of these matrix elements has been implemented in {\tt
  pw2wannier90}, the interface routine between {\tt pwscf} and {\tt
  wannier90}. To compute them, add the line
%
{\tt
\begin{quote}
write\_uHu = .true.
\end{quote}
}
%
to the input file {\tt seedname.pw2wan}.

A note on the implementation: the optical conductivity and orbital
magnetization are implemented in the {\tt berry} module in the manner
described in Refs.~\cite{yates-prb07} and \cite{lopez-prb12}
respectively. As for the AHC, it is currently implemented using the
``trace formulation'' of Ref.~\cite{lopez-prb12}, rather than the
original formulation of Ref.~\cite{wang-prb06} (the two are
equivalent, and should produce identical results to within numerical
accuracy).
