\chapter{Electronic transport calculations with the \bw\ module}\label{ch:boltzwann}

By setting $\verb#boltzwann#=\verb#TRUE#$, \postw\ will call the \bw\ routines to calculate some transport coefficients using the Boltzmann transport equation in the relaxation time approximation.

In particular, the transport coefficients that are calculated are: the electrical conductivity $\bvec \sigma$, the Seebeck coefficient $\bvec S$ and the coefficient $\bvec K$ (defined below; it is the main ingredient of the thermal conductivity). 

In the present version, the code is implemented for bulk materials only, with periodicity along all three spatial directions.

The list of parameters of the \bw\ module are summarized in Table~\ref{parameter_keywords_bw}. 
An example of a Boltzmann transport calculation can be found in the \wannier\ Tutorial. 

\textbf{Note}: if you are interested in studying 2D systems, set the correct value for the \texttt{boltz\_2d\_dir} variable (see Sec.~\ref{sec:boltz2ddir} for the documentation).

%Reference: [BoltzWann paper]
\section{Theory}
\label{sec:boltzwann-theory}
The theory of the electronic transport using the Boltzmann transport equations can be found for instance in Refs.~\cite{Ziman,Grosso,Mahan}. Here we briefly summarize only the main results. 

The current density $\bvec J$ and the heat current (or energy flux density) $\bvec J_Q$ can be written, respectively, as
\begin{align}
  \bvec J   &= \bvec \sigma(\bvec E - \bvec S \bvec \nabla T) \\
  \bvec J_Q &= T \bvec \sigma \bvec S \bvec E - \bvec K \bvec \nabla T,
\end{align}
where the electrical conductivity $\bvec \sigma$, the Seebeck coefficient $\bvec S$ and $\bvec K$ are $3\times 3$ tensors, in general.

Note: the thermal conductivity $\bvec \kappa$ (actually, the electronic part of the thermal conductivity), which is defined as the heat current per unit of temperature gradient in open-circuit experiments (i.e., with $\bvec J=0$) is not precisely $\bvec K$, but  $\bvec\kappa = \bvec K-\bvec S \bvec \sigma \bvec S T$ (see for instance Eq.~(7.89) of Ref.~\cite{Ziman} or Eq.~(XI-57b) of Ref.~\cite{Grosso}).
The thermal conductivity $\bvec \kappa$ can be then calculated from the $\bvec \sigma$, $\bvec S$ and $\bvec K$ tensors output by the code.

These quantities depend on the value of the chemical potential $\mu$ and on the temperature $T$, and can be calculated as follows:
\begin{align}
  [\bvec \sigma]_{ij}(\mu,T)&=e^2 \int_{-\infty}^{+\infty} d\varepsilon \left(-\frac {\partial f(\varepsilon,\mu,T)}{\partial \varepsilon}\right)\Sigma_{ij}(\varepsilon), \\
  [\bvec \sigma \bvec S]_{ij}(\mu,T)&=\frac e T \int_{-\infty}^{+\infty} d\varepsilon \left(-\frac {\partial f(\varepsilon,\mu,T)}{\partial \varepsilon}\right)(\varepsilon-\mu)\Sigma_{ij}(\varepsilon), \\
  [\bvec K]_{ij}(\mu,T)&=\frac 1 T \int_{-\infty}^{+\infty} d\varepsilon \left(-\frac {\partial f(\varepsilon,\mu,T)}{\partial \varepsilon}\right)(\varepsilon-\mu)^2 \Sigma_{ij}(\varepsilon),\label{eq:boltz-thermcond}
\end{align}
where $[\bvec \sigma \bvec S]$ denotes the product of the two tensors $\bvec \sigma$ and $\bvec S$, $f(\varepsilon,\mu,T)$ is the usual Fermi--Dirac distribution function 
\begin{equation*}
  f(\varepsilon,\mu,T) = \frac{1}{e^{(\varepsilon-\mu)/K_B T}+1}
\end{equation*}
and $\Sigma_{ij}(\varepsilon)$ is the Transport Distribution Function (TDF) tensor, defined as
\begin{equation*}
  \Sigma_{ij}(\varepsilon) = \frac 1 V \sum_{n,\bvec k} v_i(n,\bvec k) v_j(n,\bvec k) \tau(n,\bvec k) \delta(\varepsilon - E_{n,k}).
\end{equation*}

In the above formula, the sum is over all bands $n$ and all states $\bvec k$ (including spin, even if the spin index is not explicitly written here). $E_{n,\bvec k}$ is the energy of the $n-$th band at $\bvec k$, $v_i(n,\bvec k)$ is the $i-$th component of the band velocity at $(n,\bvec k)$, $\delta$ is the Dirac's delta function, $V$ is the cell volume, and finally $\tau$ is the relaxation time. In the \emph{relaxation-time approximation} adopted here, $\tau$ is assumed as a constant, i.e., it is independent of $n$ and $\bvec k$ and its value (in fs) is read from the input variable \verb#boltz_relax_time#.

\section{Files}
\subsection{{\tt seedname\_boltzdos.dat}}
OUTPUT. Written by \postw\ if {\tt boltz\_calc\_also\_dos} is \verb#true#. Note that even if there are other general routines in \postw\ which specifically calculate the DOS, it may be convenient to use the routines in \bw\ setting {\tt boltz\_calc\_also\_dos = true} if one must also calculate the transport coefficients. In this way, the (time-demanding) band interpolation on the $k$ mesh is performed only once, resulting in a much shorter execution time.

The first lines are comments (starting with \# characters) which describe the content of the file.
Then, there is a line for each energy $\varepsilon$ on the grid, containing a number of columns. The first column is the energy $\varepsilon$. The following is the DOS at the given energy $\varepsilon$.
The DOS can either be calculated using the adaptive smearing scheme\footnote{%
Note that in \bw\ the adaptive (energy) smearing scheme also implements a simple adaptive $k-$mesh scheme:
if at any given $k$ point one of the band gradients is zero, then that $k$ point is replaced by 8 neighboring $k$ points. Thus, the final results for the DOS may be slightly different with respect to that given by the {\tt dos} module.} if {\tt boltz\_dos\_adpt\_smr} is \verb#true#, or using a ``standard'' fixed smearing, whose type and value are defined by {\tt boltz\_dos\_smr\_type} and {\tt boltz\_dos\_smr\_fixed\_en\_width}, respectively.
If spin decomposition is required (input flag {\tt spin\_decomp}), further columns are printed, with the spin-up projection of the DOS, followed by spin-down projection.

\subsection{{\tt seedname\_tdf.dat}}
OUTPUT. This file contains the Transport Distribution Function (TDF) tensor $\bvec \Sigma$ on a grid of energies. 

The first lines are comments (starting with \# characters) which describe the content of the file.
Then, there is a line for each energy $\varepsilon$ on the grid, containing a number of columns. The first is the energy $\varepsilon$, the followings are the components if $\bvec \Sigma(\varepsilon)$ in the following order: $\Sigma_{xx}$, $\Sigma_{xy}$, $\Sigma_{yy}$, $\Sigma_{xz}$, $\Sigma_{yz}$, $\Sigma_{zz}$. If spin decomposition is required (input flag {\tt spin\_decomp}), 12 further columns are provided, with the 6 components of $\bvec \Sigma$ for the spin up, followed by those for the spin down.

The energy $\varepsilon$ is in eV, while $\bvec \Sigma$ is in 
 $\displaystyle\frac{1}{\hbar^2}\cdot\frac{\mathrm{eV}\cdot\mathrm{fs}}{\mathrm{\AA}}$.

\subsection{{\tt seedname\_elcond.dat}}
OUTPUT. This file contains the electrical conductivity tensor $\bvec \sigma$ on the grid of $T$ and $\mu$ points. 

The first lines are comments (starting with \# characters) which describe the content of the file.
Then, there is a line for each $(\mu,T)$ pair, containing 8 columns, which are respectively: $\mu$, $T$, $\sigma_{xx}$, $\sigma_{xy}$, $\sigma_{yy}$, $\sigma_{xz}$, $\sigma_{yz}$, $\sigma_{zz}$.

The chemical potential is in eV, the temperature is in K, and the components of the electrical conductivity tensor ar in SI units, i.e. in 1/$\Omega$/m.
\subsection{{\tt seedname\_seebeck.dat}}
OUTPUT. This file contains the Seebeck tensor $\bvec S$ on the grid of $T$ and $\mu$ points. 

Note that in the code the Seebeck coefficient is defined as zero when the determinant of the electrical conductivity $\bvec \sigma$ is zero. If there is at least one $(\mu, T)$ pair for which $\det \bvec \sigma=0$, a warning is issued on the output file.

The first lines are comments (starting with \# characters) which describe the content of the file.
Then, there is a line for each $(\mu,T)$ pair, containing 8 columns, which are respectively: $\mu$, $T$, $S_{xx}$, $S_{xy}$, $S_{yy}$, $S_{xz}$, $S_{yz}$, $S_{zz}$.

The chemical potential is in eV, the temperature is in K, and the components of the Seebeck tensor ar in SI units, i.e. in V/K.

\subsection{{\tt seedname\_kappa.dat}}
OUTPUT. This file contains the tensor $\bvec K$ defined in Sec.~\ref{sec:boltzwann-theory} on the grid of $T$ and $\mu$ points.

The first lines are comments (starting with \# characters) which describe the content of the file.
Then, there is a line for each $(\mu,T)$ pair, containing 8 columns, which are respectively: $\mu$, $T$, $K_{xx}$, $K_{xy}$, $K_{yy}$, $K_{xz}$, $K_{yz}$, $K_{zz}$.

The chemical potential is in eV, the temperature is in K, and the components of the $\bvec K$ tensor are the SI units for the thermal conductivity, i.e. in W/m/K.





