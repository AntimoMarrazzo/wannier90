\chapter{Files}


\section{{\tt seedname.win}}
INPUT. The master input file; contains the specification of the system
and any parameters for the run. For a description of input parameters,
see Chapter~\ref{chap:parameters}; for examples, see
Section~\ref{winfile} and the \wannier\
Tutorial.

\subsection{Units}

The following are the dimensional quantities that are
specified in the master input file:

\begin{itemize}
\item Direct lattice vectors
\item Positions (of atomic or projection) centres in real space
\item Energy windows
\item Positions of k-points in reciprocal space
\item Convergence thresholds for the minimisation of $\Omega$
%%\item \verb#zona# and \verb#box-size# (see Section~\ref{sec:proj})
\item \verb#zona# (see Section~\ref{sec:proj})
\item \verb#wannier_plot_cube#: cut-off radius for plotting WF in
  Gaussian cube format
\end{itemize}

Notes:

\begin{itemize}
\item The units (either \verb#ang#
  (default) or \verb#bohr#) in which the lattice vectors, atomic
  positions or projection centres are given can be set in the first
  line of the blocks 
  \verb#unit_cell_cart#, \verb#atoms_cart# and \verb#projections#,
  respectively, in \verb#seedname.win#.
\item Energy is always in eV.
\item Convergence thresholds are always in \AA$^{2}$
\item Positions of k-points are always in crystallographic
  coordinates relative to the reciprocal lattice vectors.
%%\item \verb#box-size# and \verb#zona# always in Angstrom and
%%  reciprocal Angstrom, respectively
\item \verb#zona# is always in reciprocal Angstrom (\AA$^{-1}$)
\item The keyword \verb#length_unit# may be set to \verb#ang#
  (default) or \verb#bohr#, in order to set the units in which the
  quantities in the output file {\tt seedname.wout} are written.
\item \verb#wannier_plot_radius# is in Angstrom
\end{itemize}

The reciprocal lattice vectors
$\{\mathbf{B}_{1},\mathbf{B}_{2},\mathbf{B}_{3}\}$ are defined in
terms
of the direct lattice vectors
$\{\mathbf{A}_{1},\mathbf{A}_{2},\mathbf{A}_{3}\}$ by the equation

\begin{equation}
\mathbf{B}_{1} = \frac{2\pi}{\Omega}\mathbf{A}_{2}\times\mathbf{A}_{3}
\ \ \ \mathrm{etc.},
\end{equation}

where the cell volume is
$V=\mathbf{A}_{1}\cdot(\mathbf{A}_{2}\times\mathbf{A}_{3})$.

\section{{\tt seedname.mmn}}
INPUT. Written by the underlying electronic structure code. See
Chapter~\ref{ch:wann-pp} for details.

\section{{\tt seedname.amn}}
INPUT. Written by the underlying electronic structure code. See
Chapter~\ref{ch:wann-pp} for details. 

\section{{\tt seedname.eig}}
INPUT. Written by the underlying electronic structure code. See
Chapter~\ref{ch:wann-pp} for details.

\section{{\tt seedname.nnkp}} \label{sec:old-nnkp}
OUTPUT. Written by \wannier\ when {\tt postproc\_setup=.TRUE.} (or,
alternatively, when \wannier\ is run with the {\tt -pp} command-line
option). See Chapter~\ref{ch:wann-pp} for details.

\section{{\tt seedname.wout}}
OUTPUT. The master output file. Here we give a description of the main
features of the output. The verbosity of the output is controlled by
the input parameter {\tt iprint}. The higher the value, the more
detail is given in the output file. The default value is 1, which prints
minimal information.

\subsection{Header}

The header provides some basic information about \wannier, the
authors, and the execution time of the current run.

\begin{verbatim}

             +---------------------------------------------------+
             |                                                   |
             |                   WANNIER90                       |
             |                                                   |
             +---------------------------------------------------+
             |                                                   |
             |        Welcome to the Maximally-Localized         |
             |        Generalized Wannier Functions code         |
             |            http://www.wannier.org                 |
             |                                                   |
             |  Wannier90 v2.0 Authors:                          |
             |    Arash A. Mostofi  (Imperial College London)    |
             |    Giovanni Pizzi    (EPFL)                       |
             |    Ivo Souza         (Universidad del Pais Vasco) |
             |    Jonathan R. Yates (University of Oxford)       |
             |                                                   |
             |  Wannier90 Contributors:                          |
             |    Young-Su Lee       (KIST, S. Korea)            |
             |    Matthew Shelley    (Imperial College London)   |
             |    Nicolas Poilvert   (Penn State University)     |
             |    Raffaello Bianco   (Paris 6 and CNRS)          |
             |    Gabriele Sclauzero (ETH Zurich)                |
             |                                                   |
             |  Wannier77 Authors:                               |
             |    Nicola Marzari    (EPFL)                       |
             |    Ivo Souza         (Universidad del Pais Vasco) |
             |    David Vanderbilt  (Rutgers University)         |
             |                                                   |
                                       .
                                       .
             | Copyright (c) 1996-2015                           |
             |        Arash A. Mostofi, Jonathan R. Yates,       |
             |        Young-Su Lee, Giovanni Pizzi, Ivo Souza,   |
             |        David Vanderbilt and Nicola Marzari        |
             |                                                   |
             |        Release: 2.0.1   2nd April 2015            |
                                       .
                                       .
             |                                                   |
             +---------------------------------------------------+
             |    Execution started on  2Apr2015 at 18:39:42     |
             +---------------------------------------------------+

\end{verbatim}

\subsection{System information}

This part of the output file presents information that \wannier\ has
read or inferred from the master input file {\tt seedname.win}. This
includes real and reciprocal lattice vectors, atomic positions,
k-points, parameters for job control, disentanglement, localisation
and plotting. 

\begin{verbatim}
                                    ------
                                    SYSTEM
                                    ------
 
                              Lattice Vectors (Ang)
                    a_1     3.938486   0.000000   0.000000
                    a_2     0.000000   3.938486   0.000000
                    a_3     0.000000   0.000000   3.938486
 
                   Unit Cell Volume:      61.09251  (Ang^3)
 
                        Reciprocal-Space Vectors (Ang^-1)
                    b_1     1.595330   0.000000   0.000000
                    b_2     0.000000   1.595330   0.000000
                    b_3     0.000000   0.000000   1.595330
  
 *----------------------------------------------------------------------------*
 |   Site       Fractional Coordinate          Cartesian Coordinate (Ang)     |
 +----------------------------------------------------------------------------+
 | Ba   1   0.00000   0.00000   0.00000   |    0.00000   0.00000   0.00000    |
 | Ti   1   0.50000   0.50000   0.50000   |    1.96924   1.96924   1.96924    |
                                          .
                                          . 
 *----------------------------------------------------------------------------*
  
                                ------------
                                K-POINT GRID
                                ------------
  
             Grid size =  4 x  4 x  4      Total points =   64
  
 *---------------------------------- MAIN ------------------------------------*
 |  Number of Wannier Functions               :                 9             |
 |  Number of input Bloch states              :                 9             |
 |  Output verbosity (1=low, 5=high)          :                 1             |
 |  Length Unit                               :               Ang             |
 |  Post-processing setup (write *.nnkp)      :                 F             |
                                              .
                                              .
 *----------------------------------------------------------------------------*
\end{verbatim}

\subsection{Nearest-neighbour k-points}

This part of the output files provides information on the
$\mathrm{b}$-vectors and weights chosen to satisfy the condition of
Eq.~\ref{eq:B1}. 

\begin{verbatim}
 *---------------------------------- K-MESH ----------------------------------*
 +----------------------------------------------------------------------------+
 |                    Distance to Nearest-Neighbour Shells                    |
 |                    ------------------------------------                    |
 |          Shell             Distance (Ang^-1)          Multiplicity         |
 |          -----             -----------------          ------------         |
 |             1                   0.398833                      6            |
 |             2                   0.564034                     12            |
                                       .
                                       .
 +----------------------------------------------------------------------------+
 | The b-vectors are chosen automatically                                     |
 | The following shells are used:   1                                         |
 +----------------------------------------------------------------------------+
 |                        Shell   # Nearest-Neighbours                        |
 |                        -----   --------------------                        |
 |                          1               6                                 |
 +----------------------------------------------------------------------------+
 | Completeness relation is fully satisfied [Eq. (B1), PRB 56, 12847 (1997)]  |
 +----------------------------------------------------------------------------+
\end{verbatim}

\subsection{Disentanglement}

Then (if required) comes the part where $\omi$ is minimised to
disentangle the optimally-connected subspace of states for the
localisation procedure in the next step.

First, a summary of the energy windows that are being used is given:
\begin{verbatim}
 *------------------------------- DISENTANGLE --------------------------------*
 +----------------------------------------------------------------------------+
 |                              Energy  Windows                               |
 |                              ---------------                               |
 |                   Outer:    2.81739  to   38.00000  (eV)                   |
 |                   Inner:    2.81739  to   13.00000  (eV)                   |
 +----------------------------------------------------------------------------+
\end{verbatim}

Then, each step of the iterative minimisation of $\omi$ is reported. 
\begin{verbatim}                                   
                   Extraction of optimally-connected subspace                  
                   ------------------------------------------                  
 +---------------------------------------------------------------------+<-- DIS
 |  Iter     Omega_I(i-1)      Omega_I(i)      Delta (frac.)    Time   |<-- DIS
 +---------------------------------------------------------------------+<-- DIS
       1       3.82493590       3.66268867       4.430E-02      0.36    <-- DIS
       2       3.66268867       3.66268867       6.911E-15      0.37    <-- DIS
                                       .
                                       .
                                   
             <<<      Delta < 1.000E-10  over  3 iterations     >>>
             <<< Disentanglement convergence criteria satisfied >>>

        Final Omega_I     3.66268867 (Ang^2)

 +----------------------------------------------------------------------------+
\end{verbatim}
The first column gives the iteration number. For a description of the
minimisation procedure and expressions for $\omi^{(i)}$, see the
original paper~\cite{souza-prb01}. The procedure is considered to be
converged when the fractional difference between $\omi^{(i)}$ and
$\omi^{(i-1)}$ is less than {\tt dis\_conv\_tol} over {\tt
  dis\_conv\_window} iterations. The final column gives a running
account of the wall time (in seconds) so far. Note that at the end of
each line of output, there are the characters ``{\tt <-- DIS}''. This
enables fast searching of the output using, for example, the Unix
command {\tt grep}:

{\tt my\_shell> grep DIS wannier.wout | less}

\subsection{Wannierisation}

The next part of the input file provides information on the
minimisation of $\omt$. At each iteration, the centre and spread of
each WF is reported.

\begin{verbatim}
*------------------------------- WANNIERISE ---------------------------------*
 +--------------------------------------------------------------------+<-- CONV
 | Iter  Delta Spread     RMS Gradient      Spread (Ang^2)      Time  |<-- CONV
 +--------------------------------------------------------------------+<-- CONV
 
 ------------------------------------------------------------------------------
 Initial State
  WF centre and spread    1  (  0.000000,  1.969243,  1.969243 )     1.52435832
  WF centre and spread    2  (  0.000000,  1.969243,  1.969243 )     1.16120620
                                      .
                                      .
      0     0.126E+02     0.0000000000       12.6297685260       0.29  <-- CONV
        O_D=      0.0000000 O_OD=      0.1491718 O_TOT=     12.6297685 <-- SPRD
 ------------------------------------------------------------------------------
 Cycle:      1
  WF centre and spread    1  (  0.000000,  1.969243,  1.969243 )     1.52414024
  WF centre and spread    2  (  0.000000,  1.969243,  1.969243 )     1.16059775
                                      .
                                      .
  Sum of centres and spreads ( 11.815458, 11.815458, 11.815458 )    12.62663472
 
      1    -0.313E-02     0.0697660962       12.6266347170       0.34  <-- CONV
        O_D=      0.0000000 O_OD=      0.1460380 O_TOT=     12.6266347 <-- SPRD
 Delta: O_D= -0.4530841E-18 O_OD= -0.3133809E-02 O_TOT= -0.3133809E-02 <-- DLTA
 ------------------------------------------------------------------------------
 Cycle:      2
  WF centre and spread    1  (  0.000000,  1.969243,  1.969243 )     1.52414866
  WF centre and spread    2  (  0.000000,  1.969243,  1.969243 )     1.16052405
                                      .
                                      .
   Sum of centres and spreads ( 11.815458, 11.815458, 11.815458 )    12.62646411
 
      2    -0.171E-03     0.0188848262       12.6264641055       0.38  <-- CONV
        O_D=      0.0000000 O_OD=      0.1458674 O_TOT=     12.6264641 <-- SPRD
 Delta: O_D= -0.2847260E-18 O_OD= -0.1706115E-03 O_TOT= -0.1706115E-03 <-- DLTA
 ------------------------------------------------------------------------------
                                      .
                                      .
 ------------------------------------------------------------------------------
 Final State
  WF centre and spread    1  (  0.000000,  1.969243,  1.969243 )     1.52416618
  WF centre and spread    2  (  0.000000,  1.969243,  1.969243 )     1.16048545
                                      .
                                      .
  Sum of centres and spreads ( 11.815458, 11.815458, 11.815458 )    12.62645344
 
         Spreads (Ang^2)       Omega I      =    12.480596753
        ================       Omega D      =     0.000000000
                               Omega OD     =     0.145856689
    Final Spread (Ang^2)       Omega Total  =    12.626453441
 ------------------------------------------------------------------------------
\end{verbatim}

It looks quite complicated, but things look more simple if one uses
{\tt grep}:

{\tt my\_shell> grep CONV wannier.wout}

gives

\begin{verbatim}
 +--------------------------------------------------------------------+<-- CONV
 | Iter  Delta Spread     RMS Gradient      Spread (Ang^2)      Time  |<-- CONV
 +--------------------------------------------------------------------+<-- CONV
      0     0.126E+02     0.0000000000       12.6297685260       0.29  <-- CONV
      1    -0.313E-02     0.0697660962       12.6266347170       0.34  <-- CONV
                                                   .
                                                   .
     50     0.000E+00     0.0000000694       12.6264534413       2.14  <-- CONV
\end{verbatim}

The first column is the iteration number, the second is the change in
$\Omega$ from the previous iteration, the third is the root-mean-squared
gradient of $\Omega$ with respect to variations in the unitary
matrices $\mathbf{U}^{(\mathbf{k})}$, and the last is the time taken (in
seconds). Depending on the input parameters used, the procedure either
runs for {\tt num\_iter} iterations, or a convergence criterion is
applied on $\Omega$. See Section~\ref{sec:wann_params} for details.

Similarly, the command

{\tt my\_shell> grep SPRD wannier.wout}

gives

\begin{verbatim}
        O_D=      0.0000000 O_OD=      0.1491718 O_TOT=     12.6297685 <-- SPRD
        O_D=      0.0000000 O_OD=      0.1460380 O_TOT=     12.6266347 <-- SPRD
                                            .
                                            .
        O_D=      0.0000000 O_OD=      0.1458567 O_TOT=     12.6264534 <-- SPRD         
\end{verbatim}

which, for each iteration, reports the value of the diagonal and
off-diagonal parts of the non-gauge-invariant spread, as well as the
total spread, respectively. Recall from Section~\ref{sec:method} that
$\Omega = \omi + \Omega_{\mathrm{D}} + \Omega_{\mathrm{OD}}$. 

\subsection{Plotting}

After WF have been localised, \wannier\ enters its plotting routines
(if required). For example, if you have specified an interpolated
bandstucture: 

\begin{verbatim}
 *---------------------------------------------------------------------------*
 |                               PLOTTING                                    |
 *---------------------------------------------------------------------------*
  
 Calculating interpolated band-structure
\end{verbatim}

\subsection{Summary timings}

At the very end of the run, a summary of the time taken for various
parts of the calculation is given. The level of detail is controlled
by the {\tt timing\_level} input parameter (set to 1 by default).

\begin{verbatim}
 *===========================================================================*
 |                             TIMING INFORMATION                            |
 *===========================================================================*
 |    Tag                                                Ncalls      Time (s)|
 |---------------------------------------------------------------------------|
 |kmesh: get                                        :         1         0.212|
 |overlap: read                                     :         1         0.060|
 |wann: main                                        :         1         1.860|
 |plot: main                                        :         1         0.168|
 *---------------------------------------------------------------------------*
 
 All done: wannier90 exiting
\end{verbatim}



\section{{\tt seedname.chk}}
INPUT/OUTPUT. Information required to restart the calculation or enter the
plotting phase. If we have used disentanglement this file also contains the
rectangular matrices $\bf{U}^{{\rm dis}({\bf k})}$.

%\section{{\tt seedname\_um.dat}}
%INPUT/OUTPUT. Contains $\bf{U}^{({\bf k})}$ and $\bf{M}^{(\bf{k,b})}$ (in the
%basis of the rotated Bloch states). Required to restart the calculation or enter the
%plotting phase.

\section{{\tt seedname.r2mn}}
OUTPUT.
Written if $\verb#write_r2mn#=\verb#true#$. The matrix elements
$\langle m|r^2|n\rangle$ (where $m$ and $n$ refer to MLWF)

\section{{\tt seedname\_band.dat}}
OUTPUT. Written if {\tt bands\_plot=.TRUE.}; The raw data for the
interpolated band structure.

\section{{\tt seedname\_band.gnu}}
OUTPUT. Written if {\tt bands\_plot=.TRUE.} and {\tt
  bands\_plot\_format=gnuplot}; A {\tt gnuplot} script to plot the
  interpolated band structure.

\section{{\tt seedname\_band.agr}}
OUTPUT. Written if {\tt bands\_plot=.TRUE.} and {\tt
  bands\_plot\_format=xmgrace}; A {\tt grace} file to plot the
  interpolated band structure.


\section{{\tt seedname\_band.kpt}}
OUTPUT. Written if {\tt bands\_plot=.TRUE.}; The k-points used for the
interpolated band structure, in units of the reciprocal lattice
vectors. This file can be used to generate a comparison band structure
from a first-principles code.

\section{{\tt seedname.bxsf}}
OUTPUT. Written if {\tt fermi\_surface\_plot=.TRUE.}; A Fermi surface plot file
suitable for plotting with XCrySDen.

\section{{\tt seedname\_w.xsf}}
OUTPUT. Written if {\tt wannier\_plot=.TRUE.} and {\tt
  wannier\_plot\_format=xcrysden}. Contains the {\tt
  w}$^{\mathrm{th}}$ WF in real space in a format suitable for
  plotting with XCrySDen or VMD, for example.

\section{{\tt seedname\_w.cube}}
OUTPUT. Written if {\tt wannier\_plot=.TRUE.} and {\tt
  wannier\_plot\_format=cube}. Contains the {\tt
  w}$^{\mathrm{th}}$ WF in real space in Gaussian cube format,
  suitable for plotting in XCrySDen, VMD, gopenmol etc.

\section{{\tt UNKp.s}}
INPUT. Read if \verb#wannier_plot#=\verb#.TRUE.# and used to plot the
MLWF. Read if \verb#transport_mode#=\verb#lcr# and \verb#tran_read_ht#=\verb#.FALSE.# 
for use in automated lcr transport calculations.

The periodic part of the Bloch states represented on a regular real
 space grid, indexed by k-point \verb#p# (from 1 to \verb#num_kpts#)
 and spin \verb#s# (`1' for `up', `2' for `down').

The name of the wavefunction file is assumed to have the form:

\begin{verbatim}
    write(wfnname,200) p,spin
200 format ('UNK',i5.5,'.',i1)
\end{verbatim}

The first line of each file should contain 5 integers: the number of
 grid points in each direction (\verb#ngx#, \verb#ngy# and
 \verb#ngz#), the k-point number \verb#ik# and the total number of
 bands \verb#num_band# in the file. The full file will be read by \wannier\ as:

\begin{verbatim}
read(file_unit) ngx,ngy,ngz,ik,nbnd
do loop_b=1,num_bands
  read(file_unit) (r_wvfn(nx,loop_b),nx=1,ngx*ngy*ngz)
end do
\end{verbatim}

The file can be in formatted or unformatted style, this is controlled
by the logical keyword \verb#wvfn_formatted#. 


\section{{\tt seedname\_centres.xyz}}

OUTPUT. Written if {\tt write\_xyz=.TRUE.}; xyz format
atomic structure file suitable for viewing with your favourite
visualiser ({\tt jmol}, {\tt gopenmol}, {\tt vmd}, etc.). 

\section{{\tt seedname\_hr.dat}}

OUTPUT. Written if {\tt hr\_plot=.TRUE.}. The first line gives the date and
time at which the file was created. 
The second line states the number of Wannier functions {\tt num\_wann}. The third
line gives the number of Wigner-Seitz grid-points {\tt nrpts}. The next block of 
{\tt nrpts} integers gives the degeneracy of each Wigner-Seitz grid point, with
15 entries per line.
Finally, the remaining {\tt num\_wann}$^2 \times$ {\tt nrpts} lines
each contain, respectively, the components of the vector $\mathbf{R}$
in terms of the lattice vectors $\{\mathbf{A}_{i}\}$, the indices $m$
and $n$, and the real and imaginary parts of the Hamiltonian matrix element
$H_{mn}^{(\mathbf{R})}$ in the WF basis, e.g.,

\begin{verbatim}
 Created on 24May2007 at 23:32:09                            
        20
        17
    4   1   2    1    4    1    1    2    1    4    6    1    1   1   2
    1   2
    0   0  -2    1    1   -0.001013    0.000000
    0   0  -2    2    1    0.000270    0.000000
    0   0  -2    3    1   -0.000055    0.000000
    0   0  -2    4    1    0.000093    0.000000
    0   0  -2    5    1   -0.000055    0.000000
    .
    .
    .
\end{verbatim}

\section{{\tt seedname\_qc.dat}}
OUTPUT. Written if $\verb#transport#=\verb#.TRUE.#$.
The first line gives the date and
time at which the file was created. 
In the subsequent lines, the energy value
in units of eV is written in the left column,
and the quantum conductance in units of 
$\frac{2e^2}{h}$ ($\frac{e^2}{h}$
for a spin-polarized system)
is written in the right column.

\begin{verbatim}
 ## written on 14Dec2007 at 11:30:17
   -3.000000       8.999999
   -2.990000       8.999999
   -2.980000       8.999999
   -2.970000       8.999999
    .
    .
    .
\end{verbatim}

\section{{\tt seedname\_dos.dat}}
OUTPUT. Written if $\verb#transport#=\verb#.TRUE.#$.
The first line gives the date and
time at which the file was created. 
In the subsequent lines, the energy value
in units of eV is written in the left column,
and the density of states in an arbitrary unit
is written in the right column.
 
\begin{verbatim}
 ## written on 14Dec2007 at 11:30:17
   -3.000000       6.801199
   -2.990000       6.717692
   -2.980000       6.640828
   -2.970000       6.569910
    .
    .
    .
\end{verbatim}


\section{{\tt seedname\_htB.dat}}

INPUT/OUTPUT. 
Read if 
$\verb#transport_mode#=\verb#bulk#$
and $\verb#tran_read_ht#=\verb#.TRUE.#$.
Written if $\verb#tran_write_ht#=\verb#.TRUE.#$. 
The first line gives the date and
time at which the file was created. 
The second line gives \verb#tran_num_bb#.
The subsequent lines contain 
\verb#tran_num_bb#$\times$\verb#tran_num_bb#
$H_{mn}$ matrix, where the indices
$m$ and $n$ span all \verb#tran_num_bb# WFs
located at $0^{\mathrm{th}}$ principal layer.
Then \verb#tran_num_bb# is recorded again in the new line 
followed by $H_{mn}$, where
$m^{\mathrm{th}}$ WF is 
at $0^{\mathrm{th}}$ principal layer
and $n^{\mathrm{th}}$ at $1^{\mathrm{st}}$ principal layer.
The $H_{mn}$ matrix is written in such a way that
$m$ is the fastest varying index.

\begin{verbatim}
 written on 14Dec2007 at 11:30:17
   150
   -1.737841   -2.941054    0.052673   -0.032926    0.010738   -0.009515
    0.011737   -0.016325    0.051863   -0.170897   -2.170467    0.202254
    .
    .
    .
   -0.057064   -0.571967   -0.691431    0.015155   -0.007859    0.000474
   -0.000107   -0.001141   -0.002126    0.019188   -0.686423  -10.379876
   150
    0.000000    0.000000    0.000000    0.000000    0.000000    0.000000
    0.000000    0.000000    0.000000    0.000000    0.000000    0.000000
    .
    .
    .
    0.000000    0.000000    0.000000    0.000000    0.000000   -0.001576
    0.000255   -0.000143   -0.001264    0.002278    0.000000    0.000000
\end{verbatim}

\section{{\tt seedname\_htL.dat}}

INPUT.
Read if $\verb#transport_mode#=\verb#lcr#$
and $\verb#tran_read_ht#=\verb#.TRUE.#$.
The file must be written in the same way as 
in \verb#seedname_htB.dat#.
The first line can be any comment you want.
The second line gives \verb#tran_num_ll#.
\verb#tran_num_ll# in \verb#seedname_htL.dat#
must be equal to
that in \verb#seedname.win#. 
The code will stop otherwise.

\begin{verbatim}
 Created by a WANNIER user
   105
    0.316879    0.000000   -2.762434    0.048956    0.000000   -0.016639
    0.000000    0.000000    0.000000    0.000000    0.000000   -2.809405
    .
    .
    .
    0.000000    0.078188    0.000000    0.000000   -2.086453   -0.001535
    0.007878   -0.545485  -10.525435
   105
    0.000000    0.000000    0.000315   -0.000294    0.000000    0.000085
    0.000000    0.000000    0.000000    0.000000    0.000000    0.000021
    .
    .
    .
    0.000000    0.000000    0.000000    0.000000    0.000000    0.000000
    0.000000    0.000000    0.000000
\end{verbatim}

\section{{\tt seedname\_htR.dat}}

INPUT.
Read if $\verb#transport_mode#=\verb#lcr#$
and $\verb#tran_read_ht#=\verb#.TRUE.#$
and $\verb#tran_use_same_lead#=\verb#.FALSE.#$.
The file must be written in the same way as 
in \verb#seedname_htL.dat#.
\verb#tran_num_rr# in \verb#seedname_htR.dat#
must be equal to
that in \verb#seedname.win#. 

\section{{\tt seedname\_htC.dat}}

INPUT.
Read if $\verb#transport_mode#=\verb#lcr#$
and $\verb#tran_read_ht#=\verb#.TRUE.#$.
The first line can be any comment you want.
The second line gives \verb#tran_num_cc#.
The subsequent lines contain 
\verb#tran_num_cc#$\times$\verb#tran_num_cc#
$H_{mn}$ matrix, where the indices
$m$ and $n$ span all \verb#tran_num_cc# WFs
inside the central conductor region.
\verb#tran_num_cc# in \verb#seedname_htC.dat#
must be equal to
that in \verb#seedname.win#. 

\begin{verbatim}
 Created by a WANNIER user
    99
  -10.499455   -0.541232    0.007684   -0.001624   -2.067078   -0.412188
    0.003217    0.076965    0.000522   -0.000414    0.000419   -2.122184
    .
    .
    .
   -0.003438    0.078545    0.024426    0.757343   -2.004899   -0.001632
    0.007807   -0.542983  -10.516896
\end{verbatim}

\section{{\tt seedname\_htLC.dat}}

INPUT.
Read if $\verb#transport_mode#=\verb#lcr#$
and $\verb#tran_read_ht#=\verb#.TRUE.#$.
The first line can be any comment you want.
The second line gives
\verb#tran_num_ll#
and \verb#tran_num_lc#
in the given order.
The subsequent lines contain 
\verb#tran_num_ll#$\times$\verb#tran_num_lc#
$H_{mn}$ matrix.
The index $m$ spans \verb#tran_num_ll# WFs
in the surface principal layer of semi-infinite left lead
which is in contact with the conductor region.
The index $n$ spans \verb#tran_num_lc# WFs
in the conductor region which
have a non-negligible interaction with
the WFs in the semi-infinite left lead.
Note that \verb#tran_num_lc# 
can be different from \verb#tran_num_cc#.


\begin{verbatim}
 Created by a WANNIER user
   105    99
    0.000000    0.000000    0.000000    0.000000    0.000000    0.000000
    0.000000    0.000000    0.000000    0.000000    0.000000    0.000000
    .
    .
    .
   -0.000003    0.000009    0.000290    0.000001   -0.000007   -0.000008
    0.000053   -0.000077   -0.000069
\end{verbatim}

\section{{\tt seedname\_htCR.dat}}

INPUT.
Read if $\verb#transport_mode#=\verb#lcr#$
and $\verb#tran_read_ht#=\verb#.TRUE.#$.
The first line can be any comment you want.
The second line gives
\verb#tran_num_cr#
and \verb#tran_num_rr#
in the given order.
The subsequent lines contain 
\verb#tran_num_cr#$\times$\verb#tran_num_rr#
$H_{mn}$ matrix.
The index $m$ spans \verb#tran_num_cr# WFs
in the conductor region which
have a non-negligible interaction with
the WFs in the semi-infinite right lead.
The index $n$ spans \verb#tran_num_rr# WFs
in the surface principal layer of semi-infinite right lead
which is in contact with the conductor region.
Note that \verb#tran_num_cr# 
can be different from \verb#tran_num_cc#.

\begin{verbatim}
 Created by a WANNIER user
    99   105
   -0.000180    0.000023    0.000133   -0.000001    0.000194    0.000008
   -0.000879   -0.000028    0.000672   -0.000257   -0.000102   -0.000029
    .
    .
    .
    0.000000    0.000000    0.000000    0.000000    0.000000    0.000000
    0.000000    0.000000    0.000000
\end{verbatim}

\section{{\tt seedname.unkg}}
\label{sec:files_unkg}

INPUT.
Read if $\verb#transport_mode#=\verb#lcr#$
and $\verb#tran_read_ht#=\verb#.FALSE.#$.
The first line is the number of G-vectors at which the
$\tilde{u}_{m\mathbf{k}}(\mathbf{G})$ are subsequently
printed. This number should always be 32 since 32 
specific $\tilde{u}_{m\mathbf{k}}$ are required.
The following lines contain the following in this order:
The band index $m$, a counter on the number of G-vectors,
the integer co-efficient of the G-vector components $a,b,c$
(where $\mathbf{G}=a\mathbf{b}_1+b\mathbf{b}_2+c\mathbf{b}_3$),
then the real and imaginary parts of the corresponding
$\tilde{u}_{m\mathbf{k}}(\mathbf{G})$ at the $\Gamma$-point. 
We note that the ordering in which the G-vectors and 
$\tilde{u}_{m\mathbf{k}}(\mathbf{G})$ are printed is not 
important, but the specific G-vectors are critical. The following 
example displays for a single band, the complete set of 
$\tilde{u}_{m\mathbf{k}}(\mathbf{G})$ that are required.
Note the G-vectors ($a,b,c$) needed.
 
\begin{verbatim}
      32
    1    1    0    0    0   0.4023306   0.0000000
    1    2    0    0    1  -0.0000325   0.0000000
    1    3    0    1    0  -0.3043665   0.0000000
    1    4    1    0    0  -0.3043665   0.0000000
    1    5    2    0    0   0.1447143   0.0000000
    1    6    1   -1    0   0.2345179   0.0000000
    1    7    1    1    0   0.2345179   0.0000000
    1    8    1    0   -1   0.0000246   0.0000000
    1    9    1    0    1   0.0000246   0.0000000
    1   10    0    2    0   0.1447143   0.0000000
    1   11    0    1   -1   0.0000246   0.0000000
    1   12    0    1    1   0.0000246   0.0000000
    1   13    0    0    2   0.0000338   0.0000000
    1   14    3    0    0  -0.0482918   0.0000000
    1   15    2   -1    0  -0.1152414   0.0000000
    1   16    2    1    0  -0.1152414   0.0000000
    1   17    2    0   -1  -0.0000117   0.0000000
    1   18    2    0    1  -0.0000117   0.0000000
    1   19    1   -2    0  -0.1152414   0.0000000
    1   20    1    2    0  -0.1152414   0.0000000
    1   21    1   -1   -1  -0.0000190   0.0000000
    1   22    1   -1    1  -0.0000190   0.0000000
    1   23    1    1   -1  -0.0000190   0.0000000
    1   24    1    1    1  -0.0000190   0.0000000
    1   25    1    0   -2  -0.0000257   0.0000000
    1   26    1    0    2  -0.0000257   0.0000000
    1   27    0    3    0  -0.0482918   0.0000000
    1   28    0    2   -1  -0.0000117   0.0000000
    1   29    0    2    1  -0.0000117   0.0000000
    1   30    0    1   -2  -0.0000257   0.0000000
    1   31    0    1    2  -0.0000257   0.0000000
    1   32    0    0    3   0.0000187   0.0000000
    2    1    0    0    0  -0.0000461   0.0000000
    .
    .
    .
\end{verbatim}


\section{{\tt seedname\_u.mat}}

OUTPUT. Written if {\tt u\_matrices\_plot=.TRUE.}. The first line gives the date and
time at which the file was created.
The second line states the number of kpoints {\tt num\_kpts} and the number of wannier
functions {\tt num\_wann} twice.
Separated by an empty line the kpoint and then the corresponding matrix is written
in num\_kpts blocks each occupying {\tt num\_wann * num\_wann + 1} lines.

\begin{verbatim}
 written on 15Sep2016 at 16:33:46 
           64           8           8
	 
   0.0000000000  +0.0000000000  +0.0000000000
   0.4468355787  +0.1394579978
  -0.0966033667  +0.4003934902
  -0.0007748974  +0.0011788678
  -0.0041177339  +0.0093821027
   .
   .
   .

   0.1250000000   0.0000000000  +0.0000000000
   0.4694005589  +0.0364941808
  +0.2287801742  -0.1135511138
  -0.4776782452  -0.0511719121
  +0.0142081014  +0.0006203139
   .
   .
   .
\end{verbatim}


\section{{\tt seedname\_u\_dis.mat}}

OUTPUT. Written if {\tt u\_matrices\_plot=.TRUE.} and disentangling enabled.
The first line gives the date and time at which the file was created.
The second line states the number of kpoints {\tt num\_kpts}, the number of wannier
functions {\tt num\_bands} and the number of {\tt num\_bands}.
Separated by an empty line the kpoint and then the corresponding matrix is written
in num\_kpts blocks each occupying {\tt num\_wann * num\_bands + 1} lines.

\begin{verbatim}
 written on 15Sep2016 at 16:33:46 
           64           8          16
	    
   0.0000000000  +0.0000000000  +0.0000000000
   1.0000000000  +0.0000000000
  +0.0000000000  +0.0000000000
  +0.0000000000  +0.0000000000
  +0.0000000000  +0.0000000000
   .
   .
   .
   
   0.1250000000   0.0000000000  +0.0000000000
   1.0000000000  +0.0000000000
  +0.0000000000  +0.0000000000
  +0.0000000000  +0.0000000000
  +0.0000000000  +0.0000000000
   .
   .
   .
\end{verbatim}
